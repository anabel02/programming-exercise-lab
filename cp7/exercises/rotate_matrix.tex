Implemente un método que realice sobre los elementos de una matriz cuadrada (la misma cantidad de filas que de columnas) cierta cantidad de rotaciones. El sentido de las rotaciones dependerá del signo de la cantidad de rotaciones. Si es positivo entonces las rotaciones se harán en el sentido de las manecillas del reloj, y si es negativo el sentido será en contra de las manecillas del reloj.

\subsection*{Ejemplos}
\begin{itemize}
    \item \( \text{Si se rota } A = 
        \begin{bmatrix}
        1 & 2 & 3 & 4 \\
        5 & 6 & 7 & 8 \\
        9 & 10 & 11 & 12 \\
        13 & 14 & 15 & 16 \\
        \end{bmatrix}
        \text{ 2 veces, daría como resultado }
        A = 
        \begin{bmatrix}
        16 & 15 & 14 & 13 \\
        12 & 11 & 10 & 9 \\
        8 & 7 & 6 & 5 \\
        4 & 3 & 2 & 1 \\
        \end{bmatrix}
        \)
    \item \( \text{Si se rota } A = 
        \begin{bmatrix}
        1 & 2 & 3 \\
        4 & 5 & 6 \\
        7 & 8 & 9 \\
        \end{bmatrix}
        \text{ -4 veces, daría como resultado }
        A = 
        \begin{bmatrix}
        3 & 6 & 9 \\
        2 & 5 & 8 \\
        1 & 4 & 7 \\
        \end{bmatrix}
        \)
\end{itemize}
