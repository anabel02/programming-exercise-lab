Para resolver el problema, primero calculamos la cantidad de días que han pasado desde el 1/1/1 hasta la fecha dada, luego usamos la operación resto de la división entre 7 para encontrar el día de la semana, ya que la semana tiene 7 días.

\begin{lstlisting}
public enum DayOfWeek
{
    Sunday = 0,
    Monday = 1,
    Tuesday = 2,
    Wednesday = 3,
    Thursday = 4,
    Friday = 5,
    Saturday = 6,
}

public static DayOfWeek CalculateDayOfWeek(int day, int month, int year)
{
    if (!ValidateDate(day, month, year))
        throw new Exception("Invalid date");
        
    int daysSinceStartOfEra = CalculateDaysSinceStartOfEra(day, month, year);
    
    return (DayOfWeek)(daysSinceStartOfEra % 7);
}
\end{lstlisting}

Aún falta implementar el método \textit{CalculateDaysSinceStartOfEra}. Dada una fecha en el formato $d/m/y$, podemos calcular los días desde el $1/1/1$ como la suma de:
\begin{enumerate}
    \item Los días desde el $1/1/1$ hasta el $31/12$ del año anterior, es decir, $(y-1)$.
    \item Los días desde el $1/1/y$, hasta el último día del mes anterior, es decir, $(m-1)$.
    \item Los días transcurridos desde el $1/m/y$  hasta $d/m/y$, o sea $d$.
\end{enumerate}

A continuación el código en C\#:

\begin{lstlisting}
public static int CalculateDaysSinceStartOfEra(int day, int month, int year)
{
    if (!ValidateDate(day, month, year))
        throw new Exception("Invalid date");
        
    int leapYears = (year - 1) / 4 - (year - 1) / 100 + (year - 1) / 400;
    
    int daysSinceStartOfEra = leapYears * 366 + (year - 1 - leapYears) * 365;
    
    if (month > 1)
        daysSinceStartOfEra += 31;
    if (month > 2)
        daysSinceStartOfEra += IsLeapYear(year) ? 29 : 28;
    if (month > 3)
        daysSinceStartOfEra += 31;
    if (month > 4)
        daysSinceStartOfEra += 30;
    if (month > 5)
        daysSinceStartOfEra += 31;
    if (month > 6)
        daysSinceStartOfEra += 30;
    if (month > 7)
        daysSinceStartOfEra += 31;
    if (month > 8)
        daysSinceStartOfEra += 31;
    if (month > 9)
        daysSinceStartOfEra += 30;
    if (month > 10)
        daysSinceStartOfEra += 31;
    if (month > 11)
        daysSinceStartOfEra += 30;
        
    daysSinceStartOfEra += day;
    
    return daysSinceStartOfEra;
}
\end{lstlisting}

Este código puede parecer engorroso porque necesitamos redefinir la cantidad de días en cada mes, algo que ya habíamos hecho en el método CalculateLastDayInMonth. Para simplificarlo, podemos usar un ciclo y reutilizar el método definido anteriormente.

\begin{lstlisting}
public static int CalculateDaysSinceStartOfEra(int day, int month, int year)
{
    if (!ValidateDate(day, month, year))
        throw new Exception("Invalid date");
        
    int leapYears = (year - 1) / 4 - (year - 1) / 100 + (year - 1) / 400;
    
    int daysSinceStartOfEra = leapYears * 366 + (year - 1 - leapYears) * 365;
    
    for (int i = 1; i < month; i++)
    {
        daysSinceStartOfEra += CalculateLastDayInMonth(i, year);
    }
    
    daysSinceStartOfEra += day;
    
    return daysSinceStartOfEra;
}
\end{lstlisting}

\textbf{Nota:} Para facilitar los cálculos asumimos que el calendario gregoriano estuvo en uso desde el año 1, lo cual no es cierto, por tanto los cálculos para fechas anteriores a 1582 no se realizan correctamente.

\begin{tcolorbox}
    El calendario juliano, utilizado desde el 46 a.C., tenía un error que acumuló un desfase de 10 días para el siglo XVI. Para corregirlo, el papa Gregorio XIII introdujo el calendario gregoriano, ajustando la duración del año y cambiando la regla de los años bisiestos, además, se eliminaron 10 días del calendario en octubre de 1582.
\end{tcolorbox}