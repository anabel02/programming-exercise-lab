Un punto está formado por dos enteros (coordenadas x,y). Implemente
un programa que reciba cuatro puntos de forma que los tres primeros formen un triángulo. Determine si el último punto es o no interior del triángulo.
\subsection*{Ejemplos}
\begin{itemize}
    \item Entrada: \texttt{(0, 0)}, \texttt{(4, 0)}, \texttt{(0, 3)}, \texttt{(1, 1)}\\
          Salida: \texttt{El punto (1, 1) está dentro del triángulo.}
    \item Entrada: \texttt{(0, 0)}, \texttt{(4, 0)}, \texttt{(0, 3)}, \texttt{(5, 1)}\\
          Salida: \texttt{El punto (5, 1) está fuera del triángulo.}
    \item Entrada: \texttt{(-2, -2)}, \texttt{(2, -2)}, \texttt{(0, 2)}, \texttt{(0, 0)}\\
          Salida: \texttt{El punto (0, 0) está dentro del triángulo.}
    \item Entrada: \texttt{(-2, -2)}, \texttt{(2, -2)}, \texttt{(0, 2)}, \texttt{(3, 0)}\\
          Salida: \texttt{El punto (3, 0) está fuera del triángulo.}
\end{itemize}
