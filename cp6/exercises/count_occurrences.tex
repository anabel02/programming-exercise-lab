Implemente un método que dado un texto, devuelva el número de ocurrencias de una palabra en él.

\subsection*{Ejemplos}

\begin{itemize}
    \item Para el texto: \texttt{''El agua es muy importante para la vida, entre otras razones porque con ella riegan las plantas.''}  
    y la palabra: \texttt{''agua''}, el resultado es \(1\).

    \item Para el texto: \texttt{''anana''}  
    y la palabra: \texttt{''ana''}, el resultado es \(2\).

    \item Para el texto: \texttt{''abcabcabc''}  
    y la palabra: \texttt{''abc''}, el resultado es \(3\).

    \item Para el texto: \texttt{''mississippi''}  
    y la palabra: \texttt{''issi''}, el resultado es \(2\). 
\end{itemize}

\textbf{Nota:} El método debe considerar las ocurrencias superpuestas. Por ejemplo, en el texto \texttt{anana} la palabra \texttt{''ana''} aparece dos veces:  
\[
\texttt{anana} \quad \rightarrow \quad \texttt{\underline{ana}na}, \quad \texttt{a\underline{ana}}.
\]
