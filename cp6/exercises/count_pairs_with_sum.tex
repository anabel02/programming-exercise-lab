Dado un número entero \( k \) y un arreglo \( arr \) de números distintos, determina cuántos pares \((a, b)\) existen en \( arr \) tales que \( a + b = k \).

\subsection*{Ejemplos}
\begin{itemize}
    \item Entrada: \( k = 5 \), \( arr = [1, 2, 3, 4] \) \\
          Salida: \( 2 \)\\
          Explicación: Los pares que suman \( 5 \) son: \( (1, 4), (2, 3) \).

    \item Entrada: \( k = 8 \), \( arr = [10, 15, 3, 7] \) \\
          Salida: \( 0 \)\\
          Explicación: Ningún par de números suma \( 8 \).

    \item Entrada: \( k = 13 \), \( arr = [5, 7, 1, 6, 11] \) \\
          Salida: \( 1 \)\\
          Explicación: El único par que suma \( 13 \) es: \( (7, 6) \).

    \item Entrada: \( k = 10 \), \( arr = [1, 2, 3, 4, 5, 6, 7, 8, 9] \) \\
          Salida: \( 4 \)\\
          Explicación: Los pares que suman \( 10 \) son: \( (1, 9), (2, 8), (3, 7), (4, 6) \).


    \item Entrada: \( k = 4 \), \( arr = [0, 6, 4, -2, -4] \) \\
          Salida: \( 2 \)\\
          Explicación: Los pares que suman \( 4 \) son: \( (0, 4), (6, -2) \).
\end{itemize}