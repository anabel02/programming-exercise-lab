Dado un arreglo \( arr \) de \( n \) elementos, encuentra el número total de \textit{inversiones}. Una inversión se define como un par de índices \( (i, j) \) tal que:

\[
i < j \quad \text{y} \quad arr[i] > arr[j].
\]

\subsection*{Ejemplos}
\begin{itemize}
    \item Entrada: \( arr = [2, 4, 1, 3, 5] \) \\
     Salida: \( 3 \)\\
     Explicación:  Las inversiones son:
    \(
    (2, 1), (4, 1), (4, 3)
    \)

    \item Entrada: \( arr = [5, 4, 3, 2, 1] \) \\
    Salida: \( 10 \)\\
    Explicación:  Todas las parejas \( (i, j) \) con \( i < j \) son inversiones, ya que el arreglo está completamente desordenado.

    \item Entrada: \( arr = [1, 2, 3, 4, 5] \) \\
    Salida: \( 0 \)\\
    Explicación:  El arreglo está completamente ordenado, por lo que no hay inversiones.
\end{itemize}
