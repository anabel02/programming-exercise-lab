Implemente un método que devuelva los elementos de un array de enteros por orden de cercanía a otro entero dado que llamaremos \texttt{pivote}. En caso de que existan varios elementos en el array a la misma distancia del elemento pivote, estos se deberán ordenar de menor a mayor entre ellos.

\subsection*{Ejemplos}

\begin{itemize}
    \item Para el array \([5, 3, 7, 10]\) y \texttt{pivote} = 7, los elementos del array quedarían: 
    \[
    [7,5,10,3]
    \]
    
    \item Para el array \([4, 2, 11, 8]\) y \texttt{pivote} = 7, los elementos del array quedarían:
    \[
    [8, 4, 11, 2]
    \]
    Note que el elemento pivote puede o no pertenecer al array.

    \item Para el array \([2, 10, 8, 4]\) y \texttt{pivote} = 7, los elementos del array quedarían:
    \[
    [8, 4, 10, 2]
    \]
    Note que los elementos 4 y 10 están a la misma distancia del pivote. En este caso, el orden correcto es que 4 aparezca antes de 10, ya que es menor.
\end{itemize}