Implemente un método que, dado un número entero \(k\), invierta cada subarray de longitud \(k\) dentro del array \(a\), comenzando desde posiciones múltiplo de \(k\). Si el último segmento tiene menos de \(k\) elementos, también debe invertirse.

\subsection*{Ejemplos}

\begin{itemize}
    \item \textbf{Entrada:} \(a = [1, 2, 3, 4, 5, 6, 7, 8, 9, 10], \, k = 3\)\\
    \textbf{Salida:} \([3, 2, 1, 6, 5, 4, 9, 8, 7, 10]\)\\
    \textbf{Explicación:} 
    \begin{itemize}
        \item El subarray \([1, 2, 3]\) se invierte en \([3, 2, 1]\).
        \item El subarray \([4, 5, 6]\) se invierte en \([6, 5, 4]\).
        \item El subarray \([7, 8, 9]\) se invierte en \([9, 8, 7]\).
        \item El último elemento \(10\), aunque no forma un subarray completo, también se invierte, resultando en sí mismo.
    \end{itemize}

    \item \textbf{Entrada:} \(a = [1, 2, 3, 4, 5, 6, 7], \, k = 4\)\\
    \textbf{Salida:} \([4, 3, 2, 1, 7, 6, 5]\)\\
    \textbf{Explicación:}
    \begin{itemize}
        \item El subarray \([1, 2, 3, 4]\) se invierte en \([4, 3, 2, 1]\).
        \item El subarray incompleto \([5, 6, 7]\) se invierte en \([7, 6, 5]\).
    \end{itemize}

    \item \textbf{Entrada:} \(a = [7, 14, 21, 28, 35, 42], \, k = 6\)\\
    \textbf{Salida:} \([42, 35, 28, 21, 14, 7]\)\\
    \textbf{Explicación:}
    \begin{itemize}
        \item Todo el array \([7, 14, 21, 28, 35, 42]\) forma un único subarray que se invierte completamente, resultando en \([42, 35, 28, 21, 14, 7]\).
    \end{itemize}

    \item \textbf{Entrada:} \(a = [1, 2, 3], \, k = 1\)\\
    \textbf{Salida:} \([1, 2, 3]\)\\
    \textbf{Explicación:}
    \begin{itemize}
        \item Como \(k = 1\), cada subarray tiene un solo elemento, por lo que no hay cambios.
    \end{itemize}
\end{itemize}
