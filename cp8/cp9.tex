\section{Multiplicación}
Multiplicar dos números sin utilizar operador \texttt{*}. Su solución debe hacer el menor número posible de llamados recursivos.

% \section{Palíndromo}
% Implemente una función recursiva que reciba un string y analice si este es palíndromo.

% \section{Inversión de Cadena}
% Implementar un método que invierta recursivamente un string y retorne el string invertido.

\section{Búsqueda binaria}
Dado un array ordenado de enteros, implemente el método
\texttt{bool BinarySearch(int[] array, int x)}
que realiza una búsqueda binaria recursiva en el array.


\section{Subsecuencia Común Más Larga}
Implementar una función recursiva que calcule la longitud de la subsecuencia común más larga entre dos strings.

\section{Distancia entre Strings}
Implementar una función recursiva que calcule la distancia entre dos strings, entendida como la cantidad de ediciones necesarias para convertir una cadena en otra (inserciones, eliminaciones o sustituciones).

\section{Suma Decreciente}
Implemente un método que reciba un entero \( n \) y muestre en la consola todas las secuencias distintas de enteros positivos que suman \( n \). Por ejemplo, para \( n = 4 \):
\[
4, 3+1, 2+2, 2+1+1, 1+1+1+1
\]

\section{Suma No Creciente}
Implemente un método que reciba un entero \( n \) y muestre en la consola todas las secuencias no crecientes de enteros positivos que suman \( n \). Por ejemplo, para \( n = 4 \):
\[
4, 3+1, 2+2, 2+1+1, 1+1+1+1
\]

\section{Inserciones Palíndromo}
Calcular el mínimo número de inserciones necesarias para volver palíndromo una cadena de entrada.

\section{Cantidad de caminos}
Implemente el método que devuelva la cantidad de caminos que se pueden tomar para llegar desde la posición (0, 0) hasta la posición ($x$,$y$).

Un camino es una secuencia de puntos con coordenadas enteras que cumplen:
- Dos puntos consecutivos distan en una unidad.
- La secuencia de las abscisas y las ordenadas, tomadas independientemente, son no decrecientes.


\section{Máximo y mínimo}
Implemente un método que, dado un array de enteros \(a\), encuentre el máximo y el mínimo realizando \(\frac{3 n}{2}\) comparaciones o menos.

\section{Recorrido del caballo}
Se desea determinar si es posible recorrer un tablero de ajedrez de \(n \times n\) celdas utilizando un caballo, de manera que, comenzando en la celda superior izquierda (0, 0), el caballo pase exactamente una vez por cada celda. Recuerde que el movimiento del caballo consiste en desplazarse dos celdas en una dirección (horizontal o vertical) y luego una celda en una dirección ortogonal.

\begin{enumerate}[label=\alph*)]
    \item Implemente un método que devuelva si existe al menos un recorrido posible del caballo en un tablero de \(n \times n\).

    \item Implemente un método que devuelva la cantidad total de recorridos posibles que el caballo puede realizar en dicho tablero.
\end{enumerate}


\section{Representación binaria}
\begin{enumerate}[label=\alph*)]
    \item Implemente un método que convierta un número de binario a decimal.\\
    Un número binario es una representación en base 2 de un número entero. El número binario está compuesto solo por los dígitos \(0\) y \(1\), donde cada posición en el número tiene un valor que es una potencia de 2, comenzando desde la derecha (posición 0).

    Para convertir un número binario a decimal, se puede utilizar la siguiente fórmula:
    \[
    n = \sum_{i=0}^{k} b_i \cdot 2^i
    \]
    donde \(b_i\) es el \(i\)-ésimo dígito del número binario, comenzando desde la derecha, y \(k\) es el índice de la posición más significativa (la izquierda).

    \subsection*{Ejemplos}
    \begin{itemize}
        \item Entrada: \( \text{1101} \) \\
        Salida: \( 13 \) \\
        Explicación:
        \[
        1101_2 = 1 \cdot 2^3 + 1 \cdot 2^2 + 0 \cdot 2^1 + 1 \cdot 2^0 = 8 + 4 + 0 + 1 = 13.
        \]
        El número binario \(1101_2\) equivale a \(13\) en decimal.

        \item Entrada: \( \text{10101} \) \\
        Salida: \( 21 \) \\
        Explicación:
        \[
        10101_2 = 1 \cdot 2^4 + 0 \cdot 2^3 + 1 \cdot 2^2 + 0 \cdot 2^1 + 1 \cdot 2^0 = 16 + 0 + 4 + 0 + 1 = 21.
        \]
        El número binario \(10101_2\) equivale a \(21\) en decimal.
    \end{itemize}

    \item Implemente un método que reciba un número entero no negativo y devuelva un string con su representación binaria.\\
    El número binario correspondiente se obtiene dividiendo el número entre 2 repetidamente, y registrando los restos de cada división. El número binario es el conjunto de los restos en orden inverso.

    \subsection*{Ejemplos}
    \begin{itemize}
        \item Entrada: \( 13 \) \\
        Salida: \( \text{1101} \) \\
        Explicación:
        \[
        \begin{aligned}
        13 \div 2 &= 6, \quad \text{resto } 1 \\
        6 \div 2 &= 3, \quad \text{resto } 0 \\
        3 \div 2 &= 1, \quad \text{resto } 1 \\
        1 \div 2 &= 0, \quad \text{resto } 1
        \end{aligned}
        \]
        Los restos en orden inverso son \(1101\), por lo tanto, la representación binaria de \(13\) es \(1101\).

        \item Entrada: \( 21 \) \\
        Salida: \( \text{10101} \) \\
        Explicación:
        \[
        \begin{aligned}
        21 \div 2 = 10,\quad \text{ resto } 1 \\
        10 \div 2 = 5,\quad\quad\quad \text{ resto } 0 \\
        5 \div 2 = 2,\quad\quad \text{ resto } 1 \\
        2 \div 2 = 1,\quad \text{ resto } 0 \\
        1 \div 2 = 0,\quad \text{ resto } 1
        \end{aligned}
        \]
        Los restos en orden inverso son \(10101\), por lo tanto, la representación binaria de \(21\) es \(10101\).
    \end{itemize}
\end{enumerate}


\section{Conjunto de Wirth}
 El conjunto de Wirth (W) se define como:

\[ 1 \in W \]
\[ \text{Si } x \in W \Rightarrow \{ 2x + 1 \in W, 3x + 1 \in W \} \]

Implemente un método que determine si el número \( x \) pertenece al conjunto de Wirth.

% \[ x \in W \iff \begin{cases}
% x = 1 \text{ o} \\
% \exists y \in W \text{ tal que } x = 2y + 1 \text{ o } x = 3y + 1
% \end{cases} \]

% \section{Mínimo elemento}
% Dado un array que no está necesariamente ordenado, implemente el método que busca recursivamente el menor de los elementos.


% \section{Suma de Elementos}
% Implementar un método que sume recursivamente todos los elementos de un array de enteros.

% \section{Descomposición en factores primos}
% Dado un número \( n \). Hallar su descomposición en factores primos. Dicha descomposición consiste en el producto de potencias de factores primos tal que se obtenga el número.


% \section{Suma de Elementos en Matriz}
% Implementar una función que sume recursivamente los elementos de una matriz.

% \section{Dígitos Crecientes}
% Implemente un método que reciba un entero \( n \) y muestre en la consola todos los números decimales de \( n \) dígitos cuyos dígitos son estrictamente crecientes. Por ejemplo, para \( n = 2 \):
% \[
% 12, 13, 14, ..., 79, 89
% \]