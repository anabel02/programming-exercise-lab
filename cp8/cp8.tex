\section{Factorial}
El factorial de un número $n$ (denotado como $n!$) se define como el producto de todos los números enteros positivos desde 1 hasta $n$, o sea:

\[
n! = \prod_{k=1}^{n} k
\]

o lo que es lo mismo:

\[
n! =
\begin{cases} 
1 & \text{si } n = 0 \\
n \cdot (n-1)! & \text{si } n > 0
\end{cases}
\]

Implemente un programa que reciba un número entero no negativo \(n\) de la consola y calcule el factorial de ese número.

\subsection*{Ejemplos:}
\begin{itemize}
    \item Entrada: \texttt{0}\\
          Salida: \texttt{El factorial de 0 es 1.}
    \item Entrada: \texttt{1}\\
          Salida: \texttt{El factorial de 1 es 1.}
    \item Entrada: \texttt{5}\\
          Salida: \texttt{El factorial de 5 es 120.}
    \item Entrada: \texttt{7}\\
          Salida: \texttt{El factorial de 7 es 5040.}
\end{itemize}


\section{Mínimo elemento}
Dado un array que no está necesariamente ordenado, implemente el método que busca recursivamente el menor de los elementos.


\section{Palíndromo}
Implemente una función recursiva que reciba un string y analice si este es palíndromo:
\[
\text{EsPalindromo}(s) = 
\begin{cases} 
\text{True} & \text{si } s \text{ tiene 0 o 1 carácter} \\
\text{EsPalindromo}(s[1:-1]) & \text{si } s[0] = s[-1] \\
\text{False} & \text{si } s[0] \neq s[-1]
\end{cases}
\]

\section{Cantidad de dígitos}
Implemente un método que reciba un número entero y halle su cantidad de dígitos (no usar la clase `string`).

% \[ CantidadDigitos(n) = \begin{cases}
% 1 & \text{si } n < 10 \\
% 1 + CantidadDigitos(\lfloor n / 10 \rfloor) & \text{de lo contrario}
% \end{cases} \]


\section{Secuencia de Collatz}
Escriba un programa que lea un número entero \( n \) de la consola e imprima la secuencia de Collatz para \( n \).

La secuencia de Collatz se define como:

\[ S_0 = n \]
\[ S_{n+1} = \left\{ \begin{array}{lcc}
             3 S_n + 1 ~ si ~ S_n ~ impar \\
             \\ \frac{S_n}{2} ~ si ~ S_n ~ par
             \end{array}
   \right. \]

Dicha secuencia termina cuando se alcanza el número \( 1 \).


\section{Potencia}
Dado dos enteros $a$ y $b$, calcule el resultado de $a^b$.


\section{Sucesión de Fibonacci}
Hallar el n-ésimo término de la sucesión de Fibonacci:

\[ F_0 = 1 \]
\[ F_1 = 1 \]
\[ F_n = F_{n-1} + F_{n-2} ~ para ~ n > 1 \]


\section{Descomposición en factores primos}
Dado un número \( n \). Hallar su descomposición en factores primos. Dicha descomposición consiste en el producto de potencias de factores primos tal que se obtenga el número.


\section{Máximo común divisor}
Implemente un método que halle el \textit{máximo común divisor} entre dos enteros utilizando el algoritmo de Euclides basado en el resto.

El \(mcd(a, b)\) es el mayor número \(d\) que divide a \(a\) y a \(b\) simultáneamente. Si \(a = b \cdot q + r\), donde:
\[
\begin{aligned}
q &\in \mathbb{Z}, \quad \text{(el cociente debe ser un número entero)} \\
0 &\leq r < b, \quad \text{(el residuo \(r\) es un número no negativo y estrictamente menor que \(b\))}.
\end{aligned}
\]
entonces:
\[
mcd(a, b) = mcd(b, r).
\]
Esto se debe a que cualquier divisor común de \(a\) y \(b\) también divide \(r\), y viceversa.


\[ mcd(a, b) = \begin{cases}
b & \text{si } a \mod b = 0 \\
mcd(b, a \% b) & \text{de lo contrario}
\end{cases} \]

\section{Torres de Hanoi}
¿Recuerdas el problema de las Torres de Hanoi? Este consiste en mover una pila de discos de una torre a otra, respetando las siguientes reglas:
\begin{itemize}
    \item Solo se puede mover un disco a la vez.
    \item Un disco nunca puede colocarse sobre otro más pequeño.
    \item Solo se puede usar una torre auxiliar.
\end{itemize}

Escribe un método recursivo para resolver el problema de las Torres de Hanoi, mostrando los movimientos de los discos.

\textbf{Ejemplo:}
Si tienes 3 discos, la secuencia de movimientos sería:
\begin{itemize}
    \item Mover disco 1 de A a C
    \item Mover disco 2 de A a B
    \item Mover disco 1 de C a B
    \item Mover disco 3 de A a C
    \item Mover disco 1 de B a A
    \item Mover disco 2 de B a C
    \item Mover disco 1 de A a C
\end{itemize}

\section{Representación binaria}
\begin{enumerate}[label=\alph*)]
    \item Implemente un método que convierta un número de binario a decimal. (El número binario está representado por un string compuesto de 0s y 1s).
    \item Implemente un método que reciba un número entero no negativo y devuelva un string con su representación binaria.
\end{enumerate}

\begin{enumerate}[label=\alph*)]
    \item Implemente un método que convierta un número de binario a decimal.\\
    Un número binario es una representación en base 2 de un número entero. El número binario está compuesto solo por los dígitos \(0\) y \(1\), donde cada posición en el número tiene un valor que es una potencia de 2, comenzando desde la derecha (posición 0).

    Para convertir un número binario a decimal, se puede utilizar la siguiente fórmula:
    \[
    n = \sum_{i=0}^{k} b_i \cdot 2^i
    \]
    donde \(b_i\) es el \(i\)-ésimo dígito del número binario, comenzando desde la derecha, y \(k\) es el índice de la posición más significativa (la izquierda).

    \subsection*{Ejemplos}
    \begin{itemize}
        \item Entrada: \( \text{1101} \) \\
        Salida: \( 13 \) \\
        Explicación:
        \[
        1101_2 = 1 \cdot 2^3 + 1 \cdot 2^2 + 0 \cdot 2^1 + 1 \cdot 2^0 = 8 + 4 + 0 + 1 = 13.
        \]
        El número binario \(1101_2\) equivale a \(13\) en decimal.

        \item Entrada: \( \text{10101} \) \\
        Salida: \( 21 \) \\
        Explicación:
        \[
        10101_2 = 1 \cdot 2^4 + 0 \cdot 2^3 + 1 \cdot 2^2 + 0 \cdot 2^1 + 1 \cdot 2^0 = 16 + 0 + 4 + 0 + 1 = 21.
        \]
        El número binario \(10101_2\) equivale a \(21\) en decimal.
    \end{itemize}

    \item Implemente un método que reciba un número entero no negativo y devuelva un string con su representación binaria.\\
    El número binario correspondiente se obtiene dividiendo el número entre 2 repetidamente, y registrando los restos de cada división. El número binario es el conjunto de los restos en orden inverso.

    \subsection*{Ejemplos}
    \begin{itemize}
        \item Entrada: \( 13 \) \\
        Salida: \( \text{1101} \) \\
        Explicación:
        \[
        \begin{aligned}
        13 \div 2 &= 6, \quad \text{resto } 1 \\
        6 \div 2 &= 3, \quad \text{resto } 0 \\
        3 \div 2 &= 1, \quad \text{resto } 1 \\
        1 \div 2 &= 0, \quad \text{resto } 1
        \end{aligned}
        \]
        Los restos en orden inverso son \(1101\), por lo tanto, la representación binaria de \(13\) es \(1101\).

        \item Entrada: \( 21 \) \\
        Salida: \( \text{10101} \) \\
        Explicación:
        \[
        \begin{aligned}
        21 \div 2 = 10,\quad \text{ resto } 1 \\
        10 \div 2 = 5,\quad\quad\quad \text{ resto } 0 \\
        5 \div 2 = 2,\quad\quad \text{ resto } 1 \\
        2 \div 2 = 1,\quad \text{ resto } 0 \\
        1 \div 2 = 0,\quad \text{ resto } 1
        \end{aligned}
        \]
        Los restos en orden inverso son \(10101\), por lo tanto, la representación binaria de \(21\) es \(10101\).
    \end{itemize}
\end{enumerate}


\section{Super primo}
Implemente un método que determine si un número es \texttt{super primo}. Un número es \texttt{super primo} si es primo y además al quitarle la última cifra sigue siendo \texttt{super primo}. Un número primo de una sola cifra se considera \texttt{super primo}. Ejemplo, 73 es superprimo.


\section{Conjunto de Wirth}
 El conjunto de Wirth (W) se define como:

\[ 1 \in W \]
\[ \text{Si } x \in W \Rightarrow \{ 2x + 1 \in W, 3x + 1 \in W \} \]

Implemente un método que determine si el número \( x \) pertenece al conjunto de Wirth.


\section{Paréntesis balanceados}
Implemente un método que, dado un entero n, calcule la cantidad de cadenas (de paréntesis abiertos y cerrados) balanceadas de longitud $2 \cdot n$.

Una cadena de paréntesis está balanceada si:
\begin{itemize}
    \item Cualquier subcadena que comienza en el principio tiene un número de paréntesis abiertos mayor o igual que el número de paréntesis cerrados.
    \item La cadena tiene el mismo número de paréntesis abiertos que cerrados.
\end{itemize}

\texttt{((()))} es una cadena balanceada y \texttt{())(()} no lo es.

\subsection*{Ejemplos}

Para \( n = 3 \), hay 5 cadenas:
\begin{itemize}
    \item \( ((())) \)
    \item \( (()()) \)
    \item \( (())() \)
    \item \( ()(()) \)
    \item \( ()()() \)
\end{itemize}

\section{Búsqueda binaria}
Dado un array ordenado de enteros, implemente el método
\texttt{bool BinarySearch(int[] array, int x)}
que realiza una búsqueda binaria recursiva en el array.


\section{Máximo y mínimo}
Implemente un método que, dado un array de enteros \(a\), encuentre el máximo y el mínimo. Encuentre un algoritmo que realice menos de \(\frac{3 n}{2}\) comparaciones para el ejercicio anterior.

\section{Suma de Elementos}
Implementar un método que sume recursivamente todos los elementos de una lista de enteros.

\begin{verbatim}
int SumaDeElementos(List<int> lista)
\end{verbatim}

\section{Inversión de Cadena}
Implementar un método que invierta recursivamente un string y retorne el string invertido.

\begin{verbatim}
string InversionDeCadena(string s)
\end{verbatim}

\section{Suma de Elementos en Matriz}
Implementar una función que sume recursivamente los elementos de una matriz.

\begin{verbatim}
int SumaDeElementosEnMatriz(int[][] matriz)
\end{verbatim}

\section{Suma de Dígitos}
Implementar una función que calcule recursivamente la suma de los dígitos de un entero \( n \).

\begin{verbatim}
int SumaDeDigitos(int n)
\end{verbatim}

\section{Decimal a Binario}
Implementar una función recursiva que retorne la representación en binario de un entero \( n \).

\begin{verbatim}
string DecimalABinario(int n)
\end{verbatim}

\section{Subsecuencia Común Más Larga}
Implementar una función recursiva que calcule la longitud de la subsecuencia común más larga entre dos strings.

\begin{verbatim}
int LongitudSubsecuenciaComun(string s1, string s2)
\end{verbatim}

\section{Distancia entre Strings}
Implementar una función recursiva que calcule la distancia entre dos strings, entendida como la cantidad de ediciones necesarias para convertir una cadena en otra (inserciones, eliminaciones o sustituciones).

\begin{verbatim}
int DistanciaEntreStrings(string s1, string s2)
\end{verbatim}

\section{Suma Decreciente}
Implemente un método que reciba un entero \( n \) y muestre en la consola todas las secuencias distintas de enteros positivos que suman \( n \). Por ejemplo, para \( n = 4 \):
\[
4, 3+1, 2+2, 2+1+1, 1+1+1+1
\]

\section{Suma No Creciente}
Implemente un método que reciba un entero \( n \) y muestre en la consola todas las secuencias no crecientes de enteros positivos que suman \( n \). Por ejemplo, para \( n = 4 \):
\[
4, 3+1, 2+2, 2+1+1, 1+1+1+1
\]

\section{Dígitos Crecientes}
Implemente un método que reciba un entero \( n \) y muestre en la consola todos los números decimales de \( n \) dígitos cuyos dígitos son estrictamente crecientes. Por ejemplo, para \( n = 2 \):
\[
01, 02, 03, ..., 89
\]
        
\section{Multiplicación}
Multiplicar dos números sin utilizar operador \texttt{*}. Su solución debe hacer el menor número posible de llamados recursivos.

\begin{verbatim}
long Producto(int a, int b)
\end{verbatim}

\section{Superprimo}
Determinar si un número es \texttt{SuperPrimo}. Un número es \texttt{SuperPrimo} si es primo y además al quitarle la última cifra sigue siendo \texttt{SuperPrimo}. Un número primo de una sola cifra se considera \texttt{SuperPrimo}. Ejemplo, 71 es superprimo.

\begin{verbatim}
bool EsSuperPrimo(int n)
\end{verbatim}

\section{Inserciones Palíndromo}
Calcular el mínimo número de inserciones necesarias para volver palíndromo una cadena de entrada.

\begin{verbatim}
int InsercionesPalindromo(string s)
\end{verbatim}

\section{Cadenas Balanceadas}
Determine cuántas cadenas de paréntesis balanceados con longitud $2 * n$ se pueden construir. Ejemplo, \texttt{((()))} es una cadena balanceada y \texttt{())(()} no lo es.

\begin{verbatim}
int CadenasBalanceadas(int n)
\end{verbatim}

\section{Eco invertido}
Programe un método que permita al usuario escribir en la Consola tantas líneas como quiera. Cuando el usuario de "ENTER" sin haber escrito algún texto, debe imprimir cada una de las líneas que escribió el usuario, pero en orden inverso. No debe usar array ni ninguna estructura de datos para almacenar todas las líneas de texto que ha ido escribiendo el usuario.

\begin{verbatim}
void EchoInvertido()
\end{verbatim}
        
\section{Cantidad de caminos}
Implemente el método que devuelva la cantidad de caminos que se pueden tomar para llegar desde la posición (0, 0) hasta la posición ($x$,$y$).

Un camino es una secuencia de puntos con coordenadas enteras que cumplen:
- Dos puntos consecutivos distan en una unidad.
- La secuencia de las abscisas y las ordenadas, tomadas independientemente, son no decrecientes.


\section{Recorrido del caballo}
Se desea determinar si es posible recorrer un tablero de ajedrez de \(n \times n\) celdas utilizando un caballo, de manera que, comenzando en la celda superior izquierda (0, 0), el caballo pase exactamente una vez por cada celda. Recuerde que el movimiento del caballo consiste en desplazarse dos celdas en una dirección (horizontal o vertical) y luego una celda en una dirección ortogonal.

\begin{enumerate}[label=\alph*)]
    \item Implemente un método que devuelva si existe al menos un recorrido posible del caballo en un tablero de \(n \times n\).

    \item Implemente un método que devuelva la cantidad total de recorridos posibles que el caballo puede realizar en dicho tablero.
\end{enumerate}
