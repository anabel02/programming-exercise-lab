\section{Mínimo elemento}
Dado un array que no está necesariamente ordenado, implemente el método que busca recursivamente el menor de los elementos.


\section{Palíndromo}
Implemente un método que determine si \texttt{s} es palíndromo (se lee igual al derecho que al revés). Ejemplos: \texttt{ana}, \texttt{anitalavalatina}, \texttt{zz}.

\section{Cantidad de dígitos}
Implemente un método que reciba un número entero y halle su cantidad de dígitos (no usar la clase `string`).


\section{Secuencia de Collatz}
Escriba un programa que lea un número entero \( n \) de la consola e imprima la secuencia de Collatz para \( n \).

La secuencia de Collatz se define como:

\[ S_0 = n \]
\[ S_{n+1} = \left\{ \begin{array}{lcc}
             3 S_n + 1 ~ si ~ S_n ~ impar \\
             \\ \frac{S_n}{2} ~ si ~ S_n ~ par
             \end{array}
   \right. \]

Dicha secuencia termina cuando se alcanza el número \( 1 \).


\section{Potencia}
Implemente un método que dado dos enteros $a$ y $b$, calcule el resultado de $a^b$.


% \[ PotenciaSucesiva(a, b) = \begin{cases}
% a \cdot PotenciaSucesiva(a, b-1) & \text{si } b > 0 \\
% 1 & \text{si } b = 0
% \end{cases} \]

\section{Sucesión de Fibonacci}
Implementa un método que halles el n-ésimo término de la sucesión de Fibonacci:

\[ F_0 = 1 \]
\[ F_1 = 1 \]
\[ F_n = F_{n-1} + F_{n-2} ~ para ~ n > 1 \]


\section{Descomposición en factores primos}
Dado un número \( n \). Hallar su descomposición en factores primos. Dicha descomposición consiste en el producto de potencias de factores primos tal que se obtenga el número.


\section{Máximo común divisor}
Hallar el Máximo Común Divisor entre dos enteros utilizando el algoritmo de Euclides basado en el resto. \( MCD(a, b) = MCD(b, r) \) donde \( r = a \% b \).


\[ mcd(a, b) = \begin{cases}
b & \text{si } a \mod b = 0 \\
mcd(b, a \% b) & \text{de lo contrario}
\end{cases} \]

\section{Torres de Hanoi}
¿Recuerdas el problema de las Torres de Hanoi? Este consiste en mover una pila de discos de una torre a otra, respetando las siguientes reglas:
\begin{itemize}
    \item Solo se puede mover un disco a la vez.
    \item Un disco nunca puede colocarse sobre otro más pequeño.
    \item Solo se puede usar una torre auxiliar.
\end{itemize}

Escribe un método recursivo para resolver el problema de las Torres de Hanoi, mostrando los movimientos de los discos.

\textbf{Ejemplo:}
Si tienes 3 discos, la secuencia de movimientos sería:
\begin{itemize}
    \item Mover disco 1 de A a C
    \item Mover disco 2 de A a B
    \item Mover disco 1 de C a B
    \item Mover disco 3 de A a C
    \item Mover disco 1 de B a A
    \item Mover disco 2 de B a C
    \item Mover disco 1 de A a C
\end{itemize}

\section{Representación binaria}
\begin{enumerate}[label=\alph*)]
    \item Implemente un método que convierta un número de binario a decimal.\\
    Un número binario es una representación en base 2 de un número entero. El número binario está compuesto solo por los dígitos \(0\) y \(1\), donde cada posición en el número tiene un valor que es una potencia de 2, comenzando desde la derecha (posición 0).

    Para convertir un número binario a decimal, se puede utilizar la siguiente fórmula:
    \[
    n = \sum_{i=0}^{k} b_i \cdot 2^i
    \]
    donde \(b_i\) es el \(i\)-ésimo dígito del número binario, comenzando desde la derecha, y \(k\) es el índice de la posición más significativa (la izquierda).

    \subsection*{Ejemplos}
    \begin{itemize}
        \item Entrada: \( \text{1101} \) \\
        Salida: \( 13 \) \\
        Explicación:
        \[
        1101_2 = 1 \cdot 2^3 + 1 \cdot 2^2 + 0 \cdot 2^1 + 1 \cdot 2^0 = 8 + 4 + 0 + 1 = 13.
        \]
        El número binario \(1101_2\) equivale a \(13\) en decimal.

        \item Entrada: \( \text{10101} \) \\
        Salida: \( 21 \) \\
        Explicación:
        \[
        10101_2 = 1 \cdot 2^4 + 0 \cdot 2^3 + 1 \cdot 2^2 + 0 \cdot 2^1 + 1 \cdot 2^0 = 16 + 0 + 4 + 0 + 1 = 21.
        \]
        El número binario \(10101_2\) equivale a \(21\) en decimal.
    \end{itemize}

    \item Implemente un método que reciba un número entero no negativo y devuelva un string con su representación binaria.\\
    El número binario correspondiente se obtiene dividiendo el número entre 2 repetidamente, y registrando los restos de cada división. El número binario es el conjunto de los restos en orden inverso.

    \subsection*{Ejemplos}
    \begin{itemize}
        \item Entrada: \( 13 \) \\
        Salida: \( \text{1101} \) \\
        Explicación:
        \[
        \begin{aligned}
        13 \div 2 &= 6, \quad \text{resto } 1 \\
        6 \div 2 &= 3, \quad \text{resto } 0 \\
        3 \div 2 &= 1, \quad \text{resto } 1 \\
        1 \div 2 &= 0, \quad \text{resto } 1
        \end{aligned}
        \]
        Los restos en orden inverso son \(1101\), por lo tanto, la representación binaria de \(13\) es \(1101\).

        \item Entrada: \( 21 \) \\
        Salida: \( \text{10101} \) \\
        Explicación:
        \[
        \begin{aligned}
        21 \div 2 = 10,\quad \text{ resto } 1 \\
        10 \div 2 = 5,\quad\quad\quad \text{ resto } 0 \\
        5 \div 2 = 2,\quad\quad \text{ resto } 1 \\
        2 \div 2 = 1,\quad \text{ resto } 0 \\
        1 \div 2 = 0,\quad \text{ resto } 1
        \end{aligned}
        \]
        Los restos en orden inverso son \(10101\), por lo tanto, la representación binaria de \(21\) es \(10101\).
    \end{itemize}
\end{enumerate}


\section{Super primo}
Un número es SuperPrimo si este es primo y el número que resulta de eliminar su cifra menos significativa es Primo.

%  bool EsSuperPrimo(int p)
% La función \( EsSuperPrimo(p) \) determina si el número \( p \) es SuperPrimo.

% \[ EsSuperPrimo(p) = \begin{cases}
% \text{True} & \text{si } p \text{ es primo y } p' \text{ es primo} \\
% \text{False} & \text{de lo contrario}
% \end{cases} \]
% donde \( p' \) es el número obtenido al eliminar la cifra menos significativa de \( p \).

\section{Conjunto de Wirth}
 El conjunto de Wirth (W) se define como:

\[ 1 \in W \]
\[ \text{Si } x \in W \Rightarrow \{ 2x + 1 \in W, 3x + 1 \in W \} \]

Implemente un método que determine si el número \( x \) pertenece al conjunto de Wirth.

% \[ x \in W \iff \begin{cases}
% x = 1 \text{ o} \\
% \exists y \in W \text{ tal que } x = 2y + 1 \text{ o } x = 3y + 1
% \end{cases} \]

\section{Paréntesis balanceados}
Implemente un método que, dado un entero n, calcule la cantidad de cadenas (de paréntesis abiertos y cerrados) balanceadas de longitud 2n.

Una cadena de paréntesis está balanceada si:
\begin{itemize}
    \item Cualquier subcadena que comienza en el principio tiene un número de paréntesis abiertos mayor o igual que el número de paréntesis cerrados.
    \item La cadena tiene el mismo número de paréntesis abiertos que cerrados.
\end{itemize}

\textbf{Ejemplo:}

Para \( n = 3 \), hay 5 cadenas:
\begin{itemize}
    \item \( ((())) \)
    \item \( (()()) \)
    \item \( (())() \)
    \item \( ()(()) \)
    \item \( ()()() \)
\end{itemize}

\section{Búsqueda binaria}
Dado un array ordenado de enteros, implemente el método
\texttt{bool BinarySearch(int[] array, int x)}
que realiza una búsqueda binaria recursiva en el array.


\section{Máximo y mínimo}
Implemente un método que, dado un array de enteros \(a\), encuentre el máximo y el mínimo. Encuentre un algoritmo que realice menos de \(\frac{3 n}{2}\) comparaciones para el ejercicio anterior.

% \section{Cantidad de caminos}
% Implemente el método que devuelva la cantidad de caminos que se pueden tomar para llegar desde la posición (0, 0) hasta la posición ($x$,$y$).

Un camino es una secuencia de puntos con coordenadas enteras que cumplen:
- Dos puntos consecutivos distan en una unidad.
- La secuencia de las abscisas y las ordenadas, tomadas independientemente, son no decrecientes.


% \section{Recorrido del caballo}
% Se desea determinar si es posible recorrer un tablero de ajedrez de \(n \times n\) celdas utilizando un caballo, de manera que, comenzando en la celda superior izquierda (0, 0), el caballo pase exactamente una vez por cada celda. Recuerde que el movimiento del caballo consiste en desplazarse dos celdas en una dirección (horizontal o vertical) y luego una celda en una dirección ortogonal.

\begin{enumerate}[label=\alph*)]
    \item Implemente un método que devuelva si existe al menos un recorrido posible del caballo en un tablero de \(n \times n\).

    \item Implemente un método que devuelva la cantidad total de recorridos posibles que el caballo puede realizar en dicho tablero.
\end{enumerate}

