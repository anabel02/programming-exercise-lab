Un número es SuperPrimo si este es primo y el número que resulta de eliminar su cifra menos significativa es Primo.

%  bool EsSuperPrimo(int p)
% La función \( EsSuperPrimo(p) \) determina si el número \( p \) es SuperPrimo.

% \[ EsSuperPrimo(p) = \begin{cases}
% \text{True} & \text{si } p \text{ es primo y } p' \text{ es primo} \\
% \text{False} & \text{de lo contrario}
% \end{cases} \]
% donde \( p' \) es el número obtenido al eliminar la cifra menos significativa de \( p \).