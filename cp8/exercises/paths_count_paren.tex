Implemente el método que devuelva la cantidad de caminos que se pueden tomar para llegar desde la posición (0, 0) hasta la posición ($x$,$x$).

Un camino es una secuencia de puntos con coordenadas enteras que cumple las siguientes condiciones:

\begin{itemize}
    \item \textbf{Distancia:} La distancia entre dos puntos consecutivos en el camino es exactamente una unidad.
    \item \textbf{Secuencia no decreciente:} Las coordenadas \(x\) y \(y\) deben seguir una secuencia no decreciente a lo largo del camino, es decir, en cada paso las coordenadas no pueden disminuir.
    \item \textbf{Restricción de la diagonal:} En ningún momento del camino se debe permitir que \(y\) sea mayor que \(x\), garantizando que siempre se esté debajo o en la misma línea de la diagonal \(y = x\).
\end{itemize}

\subsection*{Ejemplo}
Para llegar a la posición \((3, 3)\), existen exactamente \textbf{5 caminos válidos} que respetan la restricción de la diagonal:

\begin{itemize}
    \item \((0, 0) \to (1, 0) \to (2, 0) \to (3, 0) \to (3, 1) \to (3, 2) \to (3, 3)\),
    \item \((0, 0) \to (1, 0) \to (2, 0) \to (2, 1) \to (3, 1) \to (3, 2) \to (3, 3)\),
    \item \((0, 0) \to (1, 0) \to (1, 1) \to (2, 1) \to (3, 1) \to (3, 2) \to (3, 3)\),
    \item \((0, 0) \to (1, 0) \to (2, 0) \to (2, 1) \to (2, 2) \to (3, 2) \to (3, 3)\),
    \item \((0, 0) \to (1, 0) \to (1, 1) \to (2, 1) \to (2, 2) \to (3, 2) \to (3, 3)\).
\end{itemize}