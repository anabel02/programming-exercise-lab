Implemente un método que, dado un entero n, calcule la cantidad de cadenas (de paréntesis abiertos y cerrados) balanceadas de longitud $2 \cdot n$.

Una cadena de paréntesis está balanceada si:
\begin{itemize}
    \item Cualquier subcadena que comienza en el principio tiene un número de paréntesis abiertos mayor o igual que el número de paréntesis cerrados.
    \item La cadena tiene el mismo número de paréntesis abiertos que cerrados.
\end{itemize}

\texttt{((()))} es una cadena balanceada y \texttt{())(()} no lo es.

\subsection*{Ejemplos}

Para \( n = 3 \), hay 5 cadenas:
\begin{itemize}
    \item \( ((())) \)
    \item \( (()()) \)
    \item \( (())() \)
    \item \( ()(()) \)
    \item \( ()()() \)
\end{itemize}