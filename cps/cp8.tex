\begin{center}
    \begin{large}
    Cp 8 - It is all about The Matrix\\
    Curso \academicyear\\
    \end{large}
    \begin{figure}[h]
    	\centering
    	\includegraphics[width=0.5\linewidth]{cp7/matrix.png}
    \end{figure}
\end{center}

\section{Traza de una matriz}
Implemente un método que reciba una matriz cuadrada (un array bidimensional de tipo entero que representa una matriz algebraica) y devuelva la suma de los elementos de su diagonal.

\section{Chequear simetría}
Implemente un método que reciba una matriz (un array bidimensional de tipo entero que representa una matriz algebraica) y devuelva \textcolor{blue}{true} si es simétrica, \textcolor{blue}{false} en caso contrario.

\section{Cuadrado mágico}
Determine si una matriz cuadrada de enteros constituye un cuadrado mágico. Esto es que la suma de todas las filas, las columnas y ambas diagonales sea la misma.


\section{Suma de matrices}
Implemente un método que reciba dos matrices (arrays bidimensionales de tipo entero que representan matrices algebraicas) y devuelva su suma.

\section{Transpuesta}
Implemente un método que reciba una matriz (un array bidimensional de tipo entero que representa una matriz algebraica) y devuelva su transpuesta (intercambiar filas y columnas).

\section{Construcción de aeropuertos}
En un terreno de NxM expresado por la cantidad de filas y columnas de una matriz en metros se quiere construir un aeropuerto con área rectangular de Alto x Ancho (que debe caber dentro de la matríz anterior). En cada celda de la matriz se tiene la altura del terreno en ese lugar. El aeropuerto se quiere construir en la zona más "pareja" del terreno, es decir, que teniendo el área requerida en cuanto a cantidad de celdas por la horizontal y por la vertical la diferencia entre la máxima altura y la mínima altura sea la menor.

\textbf{Ejemplo:} La figura a continuación nos muestra en el rectángulo indicado en verde el área adecuada para una pista que tenga que ser de \(3 \times 2\). El rectángulo en rojo nos indica un área que es menos pareja.

\begin{center}
    \begin{tabular}{|l|l|l|l|l|l|}
    \hline
    \cellcolor{blue!20} 4 & \cellcolor{blue!20} 3 & \cellcolor{blue!20} 5 & \cellcolor{blue!20} 1 & \cellcolor{blue!20} 2 & \cellcolor{blue!20} 1 \\
    \hline
    \cellcolor{red!20} 2 & \cellcolor{red!20} 7 & \cellcolor{blue!20} 6 & \cellcolor{blue!20} 3 & \cellcolor{green!20} 2 & \cellcolor{green!20} 3 \\
    \hline
    \cellcolor{red!20} 3 & \cellcolor{red!20} 8 & \cellcolor{blue!20} 5 & \cellcolor{blue!20} 2 & \cellcolor{green!20} 2 & \cellcolor{green!20} 3 \\
    \hline
    \cellcolor{red!20} 5 & \cellcolor{red!20} 7 & \cellcolor{blue!20} 6 & \cellcolor{blue!20} 4 & \cellcolor{green!20} 2 & \cellcolor{green!20} 3 \\
    \hline
    \end{tabular}
\end{center}

\begin{enumerate}[label=\alph*)]
    \item Implemente un método que devuelva la mínima diferencia de alturas de un terreno válido para construir el aeropuerto.
    
    Note que en el área encerrada en verde la diferencia de alturas es \(1\) y en la otra es \(6\), de modo que si se invoca con \(3\) de alto y \(2\) de ancho, debe devolver como resultado \(1.\)
    
    \item * Haga una variante del método anterior pero que devuelva en lugar de la diferencia entre las alturas, las coordenadas del terreno adecuado dadas por la posición fila y columna de la celda esquina superior izquierda del terreno y la fila y columna de la celda esquina inferior derecha del terreno. 
    
    En este caso para la invocación del ejemplo anterior debe devolver el array: \(\{1,4,3,5\}\).
\end{enumerate}

\section{Triángulo de Pascal}
Dado un entero \( n \), se quiere construir una matriz de enteros que represente el Triángulo de Pascal hasta el nivel \( n \):
$$
{n\choose k} = {n-1\choose k-1} + {n-1\choose k}
$$
\textbf{Ejemplo:} 

Para \( n = 4\):
\[
\begin{array}{ccccccccc}
    &  &   &   & 1 &   &   \\
    &  &   & 1 &   & 1 &   \\
    &  & 1 &   & 2 &   & 1 \\
    &  1 &  & 3 &   & \textcolor{red}{3} &  & \textcolor{red}{1} \\
  1 & & 4 & & 6 & & \textcolor{green}{4} & & 1 
\end{array}
\]
Quedaría representado por la matriz:
\[
\begin{bmatrix}
    1 & 0 & 0 & 0 & 0 \\
    1 & 1 & 0 & 0 & 0 \\
    1 & 2 & 1 & 0 & 0 \\
    1 & 3 & \textcolor{red}{3} & \textcolor{red}{1} & 0 \\
    1 & 4 & 6 & \textcolor{green}{4} & 1 \\
\end{bmatrix}
\]
Cada término se calcula a partir de la suma de los términos que están justo encima de él  
(\textcolor{green}{4} = \textcolor{red}{3} + \textcolor{red}{1}). El nivel cero es solo un uno. 

\section{Cuidado con los ceros}
Implemente un método que reciba una matriz \(M\) y devuelva otra matriz \(M'\). La matriz \(M'\) se forma a partir de \(M\), manteniendo sus valores originales pero haciendo cero cualquier columna o fila que tenga algún cero.

\section{Punto raro} 
Implemente un método que reciba una matriz y devuelva, de existir, uno de sus puntos raros. Un punto raro es aquel que es el más pequeño de su fila, pero el más grande de su columna. En caso de existir alguno, devuelva una tupla con sus coordenadas. En caso de no existir, devuelva \texttt{null}.	

\section{Multiplicación de matrices}
Implemente un método que reciba dos matrices (arrays bidimensionales de tipo entero que representan matrices algebraicas) y devuelva su multiplicación. Asuma que las matrices de entrada son multiplicables.

\section{Espiral}
Implemente un método que reciba una matriz de \(n\) filas y \(m\) columnas y retorne un array de una sola dimensión, de tamaño \(n \times m\). El array resultante debe tener todos los elementos de la espiral que se forma comenzando por la primera posición y moviéndose a favor de las manecillas del reloj.

\section{Rotar matrices}
Implemente un método que realice sobre los elementos de una matriz cuadrada (la misma cantidad de filas que de columnas) cierta cantidad de rotaciones. El sentido de las rotaciones dependerá del signo de la cantidad de rotaciones. Si es positivo entonces las rotaciones se harán en el sentido de las manecillas del reloj, y si es negativo el sentido será en contra de las manecillas del reloj.

\subsection*{Ejemplos}
\begin{itemize}
    \item \( \text{Si se rota } A = 
        \begin{bmatrix}
        1 & 2 & 3 & 4 \\
        5 & 6 & 7 & 8 \\
        9 & 10 & 11 & 12 \\
        13 & 14 & 15 & 16 \\
        \end{bmatrix}
        \text{ 2 veces, daría como resultado }
        A = 
        \begin{bmatrix}
        16 & 15 & 14 & 13 \\
        12 & 11 & 10 & 9 \\
        8 & 7 & 6 & 5 \\
        4 & 3 & 2 & 1 \\
        \end{bmatrix}
        \)
    \item \( \text{Si se rota } A = 
        \begin{bmatrix}
        1 & 2 & 3 \\
        4 & 5 & 6 \\
        7 & 8 & 9 \\
        \end{bmatrix}
        \text{ -4 veces, daría como resultado }
        A = 
        \begin{bmatrix}
        3 & 6 & 9 \\
        2 & 5 & 8 \\
        1 & 4 & 7 \\
        \end{bmatrix}
        \)
\end{itemize}


\section{Submatriz}
Implemente un método que devuelva una submatriz eliminando una fila y una columna específica.

\textbf{Ejemplo:}

Para la matriz:
\[
\begin{bmatrix}
1 & 2 & 3 \\
4 & 5 & 6 \\
7 & 8 & 9 \\
\end{bmatrix}
\]
eliminando la fila 2 y la columna 3 resulta en:
\[
\begin{bmatrix}
1 & 2 \\
7 & 8 \\
\end{bmatrix}
\]

\section{Ordenando matrices}
Ordena los elementos que contiene de izquierda a derecha y de arriba hacia abajo. Es decir 
que cada fila esté ordenada y que el primero de una fila siga al último de la fila anterior.