\begin{center}
    \begin{large}
    Cp 4 - Ciclos\\
    Curso \academicyear\\
    \end{large}
    \begin{figure}[h]
    	\centering
    	\includegraphics[width=0.5\linewidth]{cp4/loops2.png}
    \end{figure}
\end{center}

% Basic
\section{Factorial}
El factorial de un número $n$ (denotado como $n!$) se define como el producto de todos los números enteros positivos desde 1 hasta $n$, o sea:

\[
n! = \prod_{k=1}^{n} k
\]

o lo que es lo mismo:

\[
n! =
\begin{cases} 
1 & \text{si } n = 0 \\
n \cdot (n-1)! & \text{si } n > 0
\end{cases}
\]

Implemente un programa que reciba un número entero no negativo \(n\) de la consola y calcule el factorial de ese número.

\subsection*{Ejemplos:}
\begin{itemize}
    \item Entrada: \texttt{0}\\
          Salida: \texttt{El factorial de 0 es 1.}
    \item Entrada: \texttt{1}\\
          Salida: \texttt{El factorial de 1 es 1.}
    \item Entrada: \texttt{5}\\
          Salida: \texttt{El factorial de 5 es 120.}
    \item Entrada: \texttt{7}\\
          Salida: \texttt{El factorial de 7 es 5040.}
\end{itemize}

\ifshowanswers
\section*{Respuesta:}
\begin{lstlisting}
public static long CalculateFactorial(int n)
{
    long result = 1;
    
    for (int i = 1; i <= n; i++)
    {
        result *= i;
    }
    
    return result;
}
\end{lstlisting}
\fi

\section{Suma de impares}
Implemente un programa que reciba un entero \(n\) e imprima la suma de los primeros \(n\) números impares.

Formalmente, los números impares son aquellos de la forma \(2k + 1\), donde \(k \in \mathbb{Z}_{\geq 0}\). La suma de los primeros \(n\) números impares se puede expresar como:

\[
S_n = \sum_{k=0}^{n-1} (2k + 1).
\]

\subsection*{Ejemplos}
\begin{itemize}
    \item Entrada: \texttt{1}\\
          Salida: \texttt{La suma de los primeros 1 números impares es 1.}
    \item Entrada: \texttt{3}\\
          Salida: \texttt{La suma de los primeros 3 números impares es 9.}
    \item Entrada: \texttt{5}\\
          Salida: \texttt{La suma de los primeros 5 números impares es 25.}
    \item Entrada: \texttt{7}\\
          Salida: \texttt{La suma de los primeros 7 números impares es 49.}
\end{itemize}

\ifshowanswers
\section*{Respuesta:}
\begin{lstlisting}
public static int SumOddNumbers(int n)
{
    int sum = 0;
    
    for (int i = 1; i < 2 * n; i += 2)
    {
        sum += i;
    }
    
    return sum;
}
\end{lstlisting}

Este código usa un ciclo para sumar los primeros \(n\) números impares. Sin embargo, podemos notar una propiedad interesante (que puedes demostrar por inducción):
\[
1 + 3 + 5 + \ldots + (2n-1) = n^2
\]
Por lo tanto, podemos optimizar nuestro código utilizando esta propiedad matemática.

\begin{lstlisting}
public static int SumOddNumbers(int n)
{
    return n * n;
}
\end{lstlisting}
\fi

\section{Mayor, menor y promedio}
Implemente un programa que lea una secuencia de números de la consola (uno por línea) hasta que se escriba una línea en blanco y de estos imprimir:
\begin{itemize}
    \item El mayor
    \item El menor
    \item Su promedio
\end{itemize}

\subsection*{Ejemplos:}
\begin{itemize}
    \item \textbf{Entrada:}
\begin{verbatim}
12
8
15
22
5
(línea en blanco)
\end{verbatim}
    \textbf{Salida:}
\begin{verbatim}
El mayor número es: 22
El menor número es: 5
El promedio es: 12.4
\end{verbatim}

    \item \textbf{Entrada:}
\begin{verbatim}
3
3
3
(línea en blanco)
\end{verbatim}
    \textbf{Salida:}
\begin{verbatim}
El mayor número es: 3
El menor número es: 3
El promedio es: 3.0
\end{verbatim}
\end{itemize}

\ifshowanswers
\section*{Respuesta:}
\begin{lstlisting}
string input;
int max = int.MinValue;
int min = int.MaxValue;
int sum = 0;
int count = 0;
Console.WriteLine("Introduce números enteros (deja la línea en blanco para finalizar):");

while (!string.IsNullOrWhiteSpace(input = Console.ReadLine()))
{
    if (int.TryParse(input, out int number))
    {
        if (number > max) max = number;
        if (number < min) min = number;
        sum += number;
        count++;
    }
    else
    {
        Console.WriteLine("Entrada inválida, introduce un número entero.");
    }
}
    
if (count == 0)
{
    Console.WriteLine("No se introdujeron números.");
}
else
{
    double average = (double)sum / count;

    Console.WriteLine($"El mayor: {max}");
    Console.WriteLine($"El menor: {min}");
    Console.WriteLine($"El promedio: {average}");
}
\end{lstlisting}
\fi

\section{Es primo}
Escribe un programa que determine si un entero es primo o no.
    
Un número entero positivo \( n \) se dice que es \textit{primo} si tiene exactamente dos divisores distintos: \( 1 \) y el propio número \( n \). Es decir, \( n \) es primo si y solo si no existen otros divisores \( d \) tal que \( 1 < d < n \) y \( d \) divide a \( n \). Formalmente, podemos escribir:
\[
n \text{ es primo} \iff  \forall d \in \mathbb{Z}^+ \, \text{se cumple que si} \, d \mid n \text{, entonces } d = 1 \text{ o } d = n.
\]

Donde \( \mathbb{Z}^+ \) representa el conjunto de los números enteros positivos, y \( d \mid n \) denota que \( d \) divide a \( n \), es decir, \( n \) es divisible por \( d \).
\subsection*{Ejemplos}
\begin{itemize}
    \item Entrada: \texttt{10}\\
          Salida: \textcolor{blue}{false}
    \item Entrada: \texttt{29}\\
          Salida: \textcolor{blue}{true}
    \item Entrada: \texttt{15}\\
          Salida: \textcolor{blue}{false}
    \item Entrada: \texttt{31}\\
          Salida: \textcolor{blue}{true}
\end{itemize}

% Intermediate
\section{n-ésimo primo}
Implemente un método que devuelva el $n$-ésimo primo de la sucesión de números primos.
\subsection*{Ejemplos}
\begin{itemize} 
    \item Entrada: \texttt{1}\\ 
    Salida: \texttt{2}
    \item Entrada: \texttt{3}\\
    Salida: \texttt{5}

    \item Entrada: \texttt{5}\\
          Salida: \texttt{11}
    
    \item Entrada: \texttt{10}\\
          Salida: \texttt{29}
    
    \item Entrada: \texttt{15}\\
          Salida: \texttt{47}
\end{itemize}

\section{Es perfecto}
Determina si un número entero positivo es perfecto. Un número entero positivo \( n \) se denomina \textit{perfecto} si la suma de sus divisores propios (excluyendo a \( n \)) es igual a \( n \). Formalmente:
\[
n \text{ es perfecto} \iff n = \sum_{\substack{d \mid n \\ d < n}} d,
\]

donde \( d \mid n \) indica que \( d \) es un divisor de \( n \), es decir, \( n \) es divisible por \( d \). 

\subsection*{Ejemplos}
\begin{itemize}
    \item Entrada \( n = 28 \)\\
    Salida: \textcolor{blue}{true}\\
    Explicación:
    \(
    \text{Divisores propios de 28: } 1, 2, 4, 7, 14, \quad \text{y} \quad 1 + 2 + 4 + 7 + 14 = 28.
    \)
    
    \item Entrada \( n = 6 \)\\
    Salida: \textcolor{blue}{true}\\
    Explicación:
    \(
    \text{Divisores propiosd de 6: } 1, 2, 3, \quad \text{y } 1 + 2 + 3 = 6.
    \)

    \item Entrada \( n = 12 \)\\
    Salida: \textcolor{blue}{false}\\
    Explicación:
    \(
    \text{Divisores propios de 12: } 1, 2, 3, 4, 6, \quad \text{y } 1 + 2 + 3 + 4 + 6 = 16 \neq 12.
    \)
\end{itemize}


\section{Es narcisista}
Determine si un número entero positivo es un número de Armstrong. Un número \( n \) se dice que es un \textit{número de Armstrong} (o \textit{número narcisista}) si cumple la siguiente propiedad:

\[
n = \sum_{i=1}^{d} d_i^d,
\]

donde:
\begin{itemize}
    \item \(d\) es el número de dígitos de \(n\),
    \item \(d_i\) representa el \(i\)-ésimo dígito de \(n\).
\end{itemize}

O sea, un número es un número de Armstrong si es igual a la suma de sus dígitos elevados a la cantidad de dígitos del número.

\subsection*{Ejemplos:}
\begin{itemize}
    \item Entrada \( n = 153 \)\\
    Salida: \textcolor{blue}{true}\\
    Explicación:
    \(
    \text{Dígitos de } 153: 1, 5, 3, \quad \text{y } 1^3 + 5^3 + 3^3 = 1 + 125 + 27 = 153.
    \)
    
    \item Entrada \( n = 9474 \)\\
    Salida: \textcolor{blue}{true}\\
    Explicación:
    \(
    \text{Dígitos de } 9474: 9, 4, 7, 4, \quad \text{y } 9^4 + 4^4 + 7^4 + 4^4 = 6561 + 256 + 2401 + 256 = 9474.
    \)

    \item Entrada \( n = 9475 \)\\
    Salida: \textcolor{blue}{false}\\
    Explicación:
    \(
    \text{Dígitos de } 9475: 9, 4, 7, 5, \quad \text{y } 9^4 + 4^4 + 7^4 + 5^4 = 6561 + 256 + 2401 + 625 = 9843 \neq 9475.
    \)
    
    \item Entrada \( n = 370 \)\\
    Salida: \textcolor{blue}{true}\\
    Explicación:
    \(
    \text{Dígitos de } 370: 3, 7, 0, \quad \text{y } 3^3 + 7^3 + 0^3 = 27 + 343 + 0 = 370.
    \)
\end{itemize}

\textbf{Propuesta:} Implemente un método que reciba un número entero y calcule su cantidad de dígitos (sin usar la clase \texttt{string}).


\section{Tabla de los productos}
Implementa un método que muestre en la consola la tabla de multiplicar del 1 al 10.


% Advanced
\section{Primo más cercano}
Implemente un método que, dado un número entero positivo \( n \), encuentre y retorne el número primo más cercano a \( n \). Si existen dos números primos equidistantes de \( n \), retorne el menor de los dos. 
\subsection*{Ejemplos}
\begin{itemize}
    \item Entrada \( n = 10 \)\\
    Salida: \texttt{11}

    \item Entrada \( n = 13 \)\\
    Salida: \texttt{13}

    \item Entrada \( n = 21 \)\\
    Salida: \texttt{19}\\
    Explicación:
    Los números primos más cercanos a \( 21 \) son \( 19 \) y \( 23 \). Como ambos están a igual distancia de \( 21 \), se retorna el menor.

    \item Entrada \( n = 2 \)\\
    Salida: \texttt{2}
\end{itemize}

\section{Descomponiendo en primos}
Implemente un método que reciba un número entero positivo \( n \) y escriba en la consola su descomposición en factores primos.
\subsection*{Ejemplos}
\begin{itemize}
    \item Entrada \( n = 28 \)\\
    Salida en consola: \( 2 \times 2 \times 7 \)

    \item Entrada \( n = 45 \)\\
    Salida en consola: \( 3 \times 3 \times 5 \)

    \item Entrada \( n = 13 \)\\
    Salida en consola: \( 13 \)
\end{itemize}

\section{Son amigos}
Implemente un método que diga si los números $a$ y $b$ son amigos. Dos números son amigos si la suma de los divisores de $a$ (sin contarlo a él) es $b$ y la suma de los divisores de $b$ (sin contarlo a él) es $a$. Por ejemplo, los números 220 y 284 son amigos pues los divisores propios de 220 son:
\[
1, 2, 4, 5, 10, 11, 20, 22, 44, 55, 110
\]
y los divisores propios de 284 son:
\[
1, 2, 4, 71, 142
\]
y se cumple que:
\[
284 = 1 + 2 + 4 + 5 + 10 + 11 + 20 + 22 + 44 + 55 + 110
\]
\[
220 = 1 + 2 + 4 + 71 + 142
\]

\item \textbf{Números amigos}\\
Dos números enteros positivos \( a \) y \( b \) se denominan \textit{amigos} si cumplen las siguientes condiciones:
\[
b = \sum_{d \mid a, d < a}d \quad \text{y} \quad a = \sum_{d \mid b, d < b} d,
\]
donde la notación \( d \mid a \) denota que \( d \) es un divisor de \( a \). 

Es decir, la suma de los divisores propios de \( a \) debe ser igual a \( b \), y la suma de los divisores propios de \( b \) debe ser igual a \( a \).

\subsection*{Ejemplos:}
\begin{itemize}
    \item \textbf{Entrada:} \( a = 220, \, b = 284 \)\\
    \textbf{Salida:} \textcolor{blue}{true}\\
    \textbf{Explicación:}
    \[
    \text{Divisores propios de } 220: 1, 2, 4, 5, 10, 11, 20, 22, 44, 55, 110, \quad \text{y su suma es } 284.
    \]
    \[
    \text{Divisores propios de } 284: 1, 2, 4, 71, 142, \quad \text{y su suma es } 220.
    \]
    Por lo tanto, \( 220 \) y \( 284 \) son números amigos.

    \item \textbf{Entrada:} \( a = 1184, \, b = 1210 \)\\
    \textbf{Salida:} \textcolor{blue}{true}\\
    \textbf{Explicación:}
    \[
    \text{Divisores propios de } 1184: 1, 2, 4, 8, 16, 32, 37, 74, 148, 296, 592, \quad \text{y su suma es } 1210.
    \]
    \[
    \text{Divisores propios de } 1210: 1, 2, 5, 10, 11, 22, 55, 110, 121, 242, 605, \quad \text{y su suma es } 1184.
    \]
    Por lo tanto, \( 1184 \) y \( 1210 \) son números amigos.

    \item \textbf{Entrada:} \( a = 30, \, b = 42 \)\\
    \textbf{Salida:} \textcolor{blue}{false}\\
    \textbf{Explicación:}
    \[
    \text{Divisores propios de } 30: 1, 2, 3, 5, 6, 10, 15, \quad \text{y su suma es } 42.
    \]
    \[
    \text{Divisores propios de } 42: 1, 2, 3, 6, 7, 14, 21, \quad \text{y su suma es } 54.
    \]
    Aunque la suma de los divisores propios de \( 30 \) es \( 42 \), no se cumple que la suma de los divisores propios de \( 42 \) sea \( 30 \). Por lo tanto, no son números amigos.
\end{itemize}


\section{Representación binaria}
\begin{enumerate}[label=\alph*)]
    \item Implemente un método que convierta un número de binario a decimal.\\
    Un número binario es una representación en base 2 de un número entero. El número binario está compuesto solo por los dígitos \(0\) y \(1\), donde cada posición en el número tiene un valor que es una potencia de 2, comenzando desde la derecha (posición 0).

    Para convertir un número binario a decimal, se puede utilizar la siguiente fórmula:
    \[
    n = \sum_{i=0}^{k} b_i \cdot 2^i
    \]
    donde \(b_i\) es el \(i\)-ésimo dígito del número binario, comenzando desde la derecha, y \(k\) es el índice de la posición más significativa (la izquierda).

    \subsection*{Ejemplos}
    \begin{itemize}
        \item Entrada: \( \text{1101} \) \\
        Salida: \( 13 \) \\
        Explicación:
        \[
        1101_2 = 1 \cdot 2^3 + 1 \cdot 2^2 + 0 \cdot 2^1 + 1 \cdot 2^0 = 8 + 4 + 0 + 1 = 13.
        \]
        El número binario \(1101_2\) equivale a \(13\) en decimal.

        \item Entrada: \( \text{10101} \) \\
        Salida: \( 21 \) \\
        Explicación:
        \[
        10101_2 = 1 \cdot 2^4 + 0 \cdot 2^3 + 1 \cdot 2^2 + 0 \cdot 2^1 + 1 \cdot 2^0 = 16 + 0 + 4 + 0 + 1 = 21.
        \]
        El número binario \(10101_2\) equivale a \(21\) en decimal.
    \end{itemize}

    \item Implemente un método que reciba un número entero no negativo y devuelva un string con su representación binaria.\\
    El número binario correspondiente se obtiene dividiendo el número entre 2 repetidamente, y registrando los restos de cada división. El número binario es el conjunto de los restos en orden inverso.

    \subsection*{Ejemplos}
    \begin{itemize}
        \item Entrada: \( 13 \) \\
        Salida: \( \text{1101} \) \\
        Explicación:
        \[
        \begin{aligned}
        13 \div 2 &= 6, \quad \text{resto } 1 \\
        6 \div 2 &= 3, \quad \text{resto } 0 \\
        3 \div 2 &= 1, \quad \text{resto } 1 \\
        1 \div 2 &= 0, \quad \text{resto } 1
        \end{aligned}
        \]
        Los restos en orden inverso son \(1101\), por lo tanto, la representación binaria de \(13\) es \(1101\).

        \item Entrada: \( 21 \) \\
        Salida: \( \text{10101} \) \\
        Explicación:
        \[
        \begin{aligned}
        21 \div 2 = 10,\quad \text{ resto } 1 \\
        10 \div 2 = 5,\quad\quad\quad \text{ resto } 0 \\
        5 \div 2 = 2,\quad\quad \text{ resto } 1 \\
        2 \div 2 = 1,\quad \text{ resto } 0 \\
        1 \div 2 = 0,\quad \text{ resto } 1
        \end{aligned}
        \]
        Los restos en orden inverso son \(10101\), por lo tanto, la representación binaria de \(21\) es \(10101\).
    \end{itemize}
\end{enumerate}

