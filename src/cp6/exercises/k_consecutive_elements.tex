Sea \( a \) un array de enteros y \( k \) un entero positivo. Implemente un método que determine si en \( a \) existen \( k \) elementos consecutivos iguales.

\subsection*{Ejemplos}

\begin{itemize}
    \item Para \( a = [1, 2, 2, 2, 3, 4] \) y \( k = 3 \), la salida debe ser \textcolor{blue}{true}, ya que \( 2 \) aparece tres veces consecutivas.
    \item Para \( a = [5, 5, 6, 6, 6, 7] \) y \( k = 4 \), la salida debe ser \textcolor{blue}{false}, ya que no hay ningún número que aparezca cuatro veces consecutivas.
    \item Para \( a = [1, 1, 1, 1, 1] \) y \( k = 5 \), la salida debe ser \textcolor{blue}{true}, ya que \( 1 \) aparece cinco veces consecutivas.
\end{itemize}