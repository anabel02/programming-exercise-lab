Escribe un programa que lea de la consola el radio \(r\) de un círculo, calcule su perímetro y su área, y muestre ambos resultados en la consola.\\   
\[
\text{Perímetro} = 2\pi r, \quad \text{Área} = \pi r^2
\]
\subsection*{Ejemplos}
\begin{itemize}
    \item Entrada: \texttt{r = 3}\\
          Salida: \texttt{Perímetro = 18.8496, Área = 28.2744}
    \item Entrada: \texttt{r = 0}\\
          Salida: \texttt{Perímetro = 0, Área = 0}
    \item Entrada: \texttt{r = 1.5}\\
          Salida: \texttt{Perímetro = 9.4248, Área = 7.0686}
\end{itemize}