Implemente un programa que reciba un número entero \( x \) de la consola y calcule su valor absoluto. El valor absoluto de un número \( x \) se define como el número sin su signo, es decir, la distancia de \( x \) al origen en la recta numérica. No utilice Math.Abs.

La función del valor absoluto \( |x| \) se define de la siguiente manera:
\[
|x| =
\begin{cases} 
x & \text{si } x \geq 0 \\
-x & \text{si } x < 0
\end{cases}
\]

\subsection*{Ejemplos}
\begin{itemize}
    \item Entrada: 5
    
    Salida: El valor absoluto de 5 es: 5

    \item Entrada: -8
    
    Salida: El valor absoluto de -8 es: 8

    \item Entrada: 0
    
    Salida: El valor absoluto de 0 es: 0
\end{itemize}
