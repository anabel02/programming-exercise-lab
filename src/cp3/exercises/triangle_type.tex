Implemente un programa que pida al usuario tres números enteros qque representen los lados de un triángulo y determine qué tipo de triángulo forman. Debe mostrar en la consola lo siguiente:
\begin{itemize}
	\item 0 si no pueden ser lados de ningún triángulo. Tres números pueden ser lados de un triángulo si cumplen la desigualdad triangular, que establece que la suma de las longitudes de dos lados siempre debe ser mayor que la longitud del tercer lado.
	\item 1 si es un triángulo escaleno. Un triángulo escaleno tiene los tres lados de diferente longitud.
	\item 2 si es un triángulo isóceles. Un triángulo isósceles tiene dos lados iguales y uno diferente.
	\item 3 si es un triángulo equilátero. Un triángulo equilátero tiene los tres lados iguales. 
\end{itemize}

\textbf{Propuesta:} Realizar la implementación utilizando \textcolor{blue}{enum}.

\subsection*{Ejemplos}
\begin{itemize}
    \item Entrada: side1 = 1, side2 = 2, side3 = 3

    Salida: 0

    \item Entrada: side1 = 3, side2 = 4, side3 = 5

    Salida: 1

    \item Entrada: side1 = 4, side2 = 4, side3 = 5

     Salida: 2

     \item Entrada: side1 = 6, side2 = 6, side3 = 6
     
     Salida: 3
\end{itemize}
