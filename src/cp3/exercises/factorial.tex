El factorial de un número $n$ (denotado como $n!$) se define como el producto de todos los números enteros positivos desde 1 hasta $n$, o sea:

\[
n! = \prod_{k=1}^{n} k
\]

o lo que es lo mismo:

\[
n! =
\begin{cases} 
1 & \text{si } n = 0 \\
n \cdot (n-1)! & \text{si } n > 0
\end{cases}
\]

Implemente un programa que reciba un número entero no negativo \(n\) de la consola y calcule el factorial de ese número.

\subsection*{Ejemplos}
\begin{itemize}
    \item Entrada: \texttt{0}\\
          Salida: \texttt{El factorial de 0 es 1.}
    \item Entrada: \texttt{1}\\
          Salida: \texttt{El factorial de 1 es 1.}
    \item Entrada: \texttt{5}\\
          Salida: \texttt{El factorial de 5 es 120.}
    \item Entrada: \texttt{7}\\
          Salida: \texttt{El factorial de 7 es 5040.}
\end{itemize}
