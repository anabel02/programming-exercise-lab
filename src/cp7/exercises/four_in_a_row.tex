\begin{enumerate}[label=\alph*)]
    \item Dada una matriz de elementos booleanos, implemente un método que determine si hay 4 valores idénticos en línea. Estos valores pueden estar alineados en dirección horizontal, vertical o diagonal.
    \item * Lo mismo que el inciso anterior pero en lugar de devolver si existe o no, devolver cuántas líneas de 4 hay.
\end{enumerate}


\textbf{Ejemplo:}

Supongamos la siguiente matriz donde los valores \texttt{true} están marcados con una ''T'' y los valores \texttt{false} con una ''F'':

\[
\begin{array}{cccccc}
F & F & T & F & T & F \\
\textcolor{green}{T} & \textcolor{green}{T}& \textcolor{green}{T} & \textcolor{green}{T} & F & \textcolor{red}{F} \\
F & \textcolor{green}{T} & F & T & \textcolor{red}{F} & T \\
T & F & \textcolor{green}{T} & \textcolor{red}{F} & T & F \\
T & T & \textcolor{red}{F} & \textcolor{green}{T} & F & F \\
\end{array}
\]

En este ejemplo, existen tres posibles líneas de 4 valores iguales consecutivos.
