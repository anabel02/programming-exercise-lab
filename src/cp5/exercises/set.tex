Implemente la clase \textcolor{cyan}{Set} para representar el comportamiento de un conjunto de enteros. Un conjunto es una colección de elementos únicos, es decir, no repetidos.

\subsection*{Funcionalidades a implementar}
\begin{itemize}
    \item \textbf{Obtener la cardinalidad}: 
    Método que devuelve el número de elementos en el conjunto.
    
    \item \textbf{Añadir un elemento}: 
    Método que agrega un nuevo elemento al conjunto. Si el elemento ya existe, no debe ser agregado de nuevo.
    
    \item \textbf{Eliminar un elemento}: 
    Método que elimina un elemento específico del conjunto. Si el elemento no existe en el conjunto, no se realiza ninguna acción.
    
    \item \textbf{Vaciar el conjunto}: 
    Método que elimina todos los elementos del conjunto, dejándolo vacío.

    \item \textbf{Pertenencia}: 
    Método que verifica si un elemento específico está presente en el conjunto (\(x \in B\)).
    
    \item \textbf{Intersección}: 
    Método que devuelve un nuevo conjunto con los elementos comunes entre dos conjuntos (\(A \cap B\)).
    
    \item \textbf{Unión}: 
    Método que devuelve un nuevo conjunto con todos los elementos de dos conjuntos (\(A \cup B\)),
    
    \item \textbf{Diferencia}: 
    Método que devuelve un nuevo conjunto con los elementos de \(A\) que no están en \(B\) (\(A - B\)).
    
    \item \textbf{Subconjunto}: 
    Método que determina si un conjunto está contenido dentro de otro (\(A \subseteq B\)).
    
    \item \textbf{Igualdad}: 
    Método que compara si dos conjuntos son iguales (\(A = B\)), es decir, si tienen exactamente los mismos elementos.
\end{itemize}

\textbf{Ejemplo de uso}

\begin{lstlisting}
Set s1 = new Set();
s1.Add(1);
s1.Add(2);
s1.Add(3);

Set s2 = new Set([2, 3, 4]);

// Obtener cardinalidad
Console.WriteLine($"Cardinalidad de s1: {s1.Cardinality}");
// Esperado: Cardinalidad de s1: 3

// Agregar un elemento
s1.Add(4);
Console.WriteLine($"Cardinalidad de s1 después de agregar 4: {s1.Cardinality}");
// Esperado: Cardinalidad de s1 después de agregar 4: 4

// Eliminar un elemento
s1.Remove(2);
Console.WriteLine($"Cardinalidad de s1 después de eliminar 2: {s1.Cardinality}");
// Esperado: Cardinalidad de s1 después de eliminar 2: 3

// Pertenencia
bool contains = s2.Contains(3);
Console.WriteLine($"¿s2 contiene 3?: {contains}");
// Esperado: ¿s2 contiene 3?: True

// Intersección
Set intersection = Set.Intersection(s1, s2);
Console.WriteLine($"Intersección de s1 y s2: [{string.Join(", ", intersection.Elements)}]");
// Esperado: Intersección de s1 y s2: [3, 4]

// Unión
Set union = Set.Union(s1, s2);
Console.WriteLine($"Unión de s1 y s2: [{string.Join(", ", union.Elements)}]");
// Esperado: Unión de s1 y s2: [1, 2, 3, 4]

// Diferencia
Set difference = Set.Difference(s2, s1);
Console.WriteLine($"Diferencia de s2 y s1: [{string.Join(", ", difference.Elements)}]");
// Esperado: Diferencia de s2 y s1: [2]

// Vaciar
s1.Clear()
Console.WriteLine($"Cardinalidad de s1 después de vaciar: {s1.Cardinality}");
// Esperado: Cardinalidad de s1 después de vaciar: 0

// Subconjunto
bool isSubset = Set.IsSubset(s1, s2);
Console.WriteLine($"¿s1 es subconjunto de s2?: {isSubset}");
// Esperado: ¿s1 es subconjunto de s2?: True

// Igualdad
bool isEqual = s1 == s2;
Console.WriteLine($"¿s1 es igual a s2?: {isEqual}");
// Esperado: ¿s1 es igual a s2?: False
\end{lstlisting}