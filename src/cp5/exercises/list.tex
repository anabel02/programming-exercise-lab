Implemente la clase \textcolor{cyan}{MyList} para representar una lista de enteros. 

\subsection*{Funcionalidades a Implementar}
\begin{itemize}
    \item \textbf{Obtener el número de elementos (\texttt{Count})}: 
     Propiedad que devuelve el número de elementos en la lista.
    
    \item \textbf{Añadir un elemento (\texttt{Add})}:
     Método para añadir un elemento a la lista.
    
    \item \textbf{Vaciar la Lista (\texttt{Clear})}:
    Método para eliminar todos los elementos de la lista.
    
    \item \textbf{Verificar contenido (\texttt{Contains})}:
    Método que comprueba si un elemento está presente en la lista.
     
    \item \textbf{Obtener el Índice de un Elemento (\texttt{IndexOf})}:
    Método que devuelve el índice de la primera aparición de un elemento específico en la lista.
    
    \item \textbf{Insertar en un Índice Específico (\texttt{Insert})}:
    Método para insertar un elemento en un índice específico de la lista.
    
    \item \textbf{Eliminar un Elemento (\texttt{Remove})}:
    Método para eliminar la primera aparición de un elemento específico de la lista.
    
    \item \textbf{Eliminar en un Índice Específico (\texttt{RemoveAt})}:
    Método para eliminar el elemento en un índice específico de la lista.
    
    \item \textbf{Acceder a un Elemento por Índice (\texttt{this[int index]})}:
    Propiedad indexadora que permite acceder a los elementos de la lista mediante su índice.
\end{itemize}

\textbf{Ejemplo de uso:}
\begin{lstlisting}
// Crear una instancia de MyList
MyList myList = new MyList();

// Añadir elementos a la lista
myList.Add(10);
myList.Add(20);
myList.Add(30);
Console.WriteLine("Count: " + myList.Count);
// Esperado: Count: 3

// Acceder a elementos por índice
Console.WriteLine("Elemento en índice 1: " + myList[1]);
// Esperado: Elemento en índice 1: 20

// Verificar si un elemento está en la lista
bool contains = myList.Contains(20);
Console.WriteLine("¿Contiene 20?: " + contains);
// Esperado: ¿Contiene 20?: True

// Obtener el índice de un elemento
int index = myList.IndexOf(30);
Console.WriteLine("Índice de 30: " + index);
// Esperado: Índice de 30: 2

// Insertar un elemento en un índice específico
myList.Insert(1, 15);
Console.WriteLine("Elemento en índice 1 después de Insert: " + myList[1]);
// Esperado: Elemento en índice 1 después de Insert: 15

// Eliminar un elemento por valor
myList.Remove(20);
Console.WriteLine("¿Contiene 20 después de Remove?: " + myList.Contains(20));
// Esperado: ¿Contiene 20 después de Remove?: False

// Eliminar un elemento por índice
myList.RemoveAt(0);
Console.WriteLine("Elemento en índice 0 después de RemoveAt: " + myList[0]);
// Esperado: Elemento en índice 0 después de RemoveAt: 15

// Vaciar la lista
myList.Clear();
Console.WriteLine("Count después de Clear: " + myList.Count);
// Esperado: Count después de Clear: 0
\end{lstlisting}
