Implemente una clase llamada \textcolor{cyan}{Date} para representar una fecha, que incluya día, mes y año. Esta clase debe permitir realizar operaciones y consultas básicas sobre fechas.

\subsection*{Funcionalidades a implementar}
\begin{itemize}
    \item \textbf{Obtener el día}: Devuelve el día correspondiente de la fecha.
    \item \textbf{Obtener el mes}: Devuelve el mes correspondiente de la fecha.
    \item \textbf{Obtener el año}: Devuelve el año correspondiente de la fecha.
    \item \textbf{Obtener el día de la semana}: Calcula y devuelve el día de la semana en el que cae la fecha (por ejemplo, Lunes, Martes).
    \item \textbf{Comparar fechas}: Permite comparar dos fechas, indicando si una fecha es anterior, igual o posterior a otra.
    \item \textbf{Avanzar un día}: Ajusta la fecha actual avanzando un día, manejando correctamente los cambios de mes y año cuando sea necesario.
    \item \textbf{Retroceder un día}: Ajusta la fecha actual retrocediendo un día, manejando correctamente los cambios de mes y año cuando sea necesario.
    \item \textbf{Calcular la diferencia en días}: Calcula y devuelve la cantidad de días entre dos fechas dadas.
    \item \textbf{Representación en cadena de caracteres}: Devuelve la fecha en formato \texttt{día/mes/año} (por ejemplo, \texttt{12/10/2024}).
\end{itemize}

\textbf{Ejemplo de uso:}
\begin{lstlisting}
Date date1 = new Date(12, 10, 2024);
Date date2 = new Date(15, 10, 2024);

// Obtener el día, mes, año y el día de la semana
int day = date1.Day;                // Esperado: 12
int month = date1.Month;            // Esperado: 10
int year = date1.Year;              // Esperado: 2024
string dayOfWeek = date1.DayOfWeek; // Esperado: "Saturday"

// Comparar fechas con operadores
if (date1 < date2)
{
    Console.WriteLine($"{date1} es anterior a {date2}");
}
else if (date1 > date2)
{
    Console.WriteLine($"{date1} es posterior a {date2}");
}
else
{
    Console.WriteLine($"{date1} es igual a {date2}");
}
// Esperado: "12/10/2024 es anterior a 15/10/2024"

// Avanzar un día
date1.AdvanceOneDay();
Console.WriteLine(date1.ToString()); // Esperado: 13/10/2024

// Retroceder un día
date2.RetreatOneDay();
Console.WriteLine(date2.ToString()); // Esperado: 14/10/2024

// Calcular la diferencia en días entre dos fechas
int daysDifference = Date.DaysBetween(date1, date2);
Console.WriteLine($"Días de diferencia: {daysDifference}"); // Esperado: 1
\end{lstlisting}