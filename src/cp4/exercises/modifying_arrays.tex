\begin{enumerate}[label=\alph*)]
    \item \textbf{Añadir un valor al final del array:} \\
    Implemente un método que reciba un valor \texttt{val} y lo añada al final del array \(a\), devolviendo un nuevo array con el valor añadido.
    
    \subsection*{Ejemplos:}
    \begin{itemize}
        \item Entrada: \(a = [1, 2, 3]\), \texttt{val} = 4 \\
        Salida: \([1, 2, 3, 4]\)
        \item Entrada: \(a = [5, 7, 9]\), \texttt{val} = 10 \\
        Salida: \([5, 7, 9, 10]\)
        \item Entrada: \(a = [1, -50, 2]\), \texttt{val} = 5 \\
        Salida: \([1, -50, 2, 5]\)
    \end{itemize}

    \item \textbf{Insertar un valor en una posición específica:} \\
    Implemente un método que, dado un entero \texttt{pos} y un valor \texttt{val}, inserte el valor \texttt{val} en la posición \texttt{pos} del array \(a\), desplazando los elementos existentes hacia la derecha, y devuelva un nuevo array con el valor insertado.
    
    \subsection*{Ejemplos}
    \begin{itemize}
        \item Entrada: \(a = [1, 3, 4]\), \texttt{pos} = 1, \texttt{val} = 2 \\
        Salida: \([1, 2, 3, 4]\)
        \item Entrada: \(a = [5, 10, 15]\), \texttt{pos} = 2, \texttt{val} = 12 \\
        Salida: \([5, 10, 12, 15]\)
        \item Entrada: \(a = [1, 2, 3, 4]\), \texttt{pos} = 0, \texttt{val} = 0 \\
        Salida: \([0, 1, 2, 3, 4]\)
    \end{itemize}

    \item \textbf{Eliminar un valor en una posición específica:} \\
    Implemente un método que, dado un entero \texttt{pos} referente a una posición del array \(a\), elimine el elemento en esa posición, y devuelva un nuevo array sin ese elemento.
    
    \subsection*{Ejemplos}
    \begin{itemize}
        \item Entrada: \(a = [1, 2, -10, 4]\), \texttt{pos} = 2 \\
        Salida: \([1, 2, 4]\)
        \item Entrada: \(a = [5, 10, 15, 20]\), \texttt{pos} = 1 \\
        Salida: \([5, 15, 20]\)
        \item Entrada: \(a = [8, 6, 7]\), \texttt{pos} = 0 \\
        Salida: \([6, 7]\)
    \end{itemize}

    \item \textbf{Eliminar la primera ocurrencia de un valor:} \\
    Implemente un método que, dado un valor \texttt{val}, elimine la primera ocurrencia de dicho valor en el array \(a\), y devuelva un nuevo array sin ese valor.
    
    \subsection*{Ejemplos}
    \begin{itemize}
        \item Entrada: \(a = [1, 2, 3, 2, 4]\), \texttt{val} = 2 \\
        Salida: \([1, 3, 2, 4]\)
        \item Entrada: \(a = [5, 5, 10, 15]\), \texttt{val} = 5 \\
        Salida: \([10, 15, 5]\)
        \item Entrada: \(a = [3, 2, 3, 4, 5]\), \texttt{val} = 3 \\
        Salida: \([2, 3, 4, 5]\)
    \end{itemize}
\end{enumerate}