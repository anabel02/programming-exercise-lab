Sea \( p(x) \) un polinomio de grado \( n \) con coeficientes enteros, tal que:
    
\[
p(x) = a_n x^n + a_{n-1} x^{n-1} + \dots + a_1 x + a_0
\]

Este polinomio puede ser representado mediante un arreglo \( A = [a_0, a_1, \dots, a_n] \), donde el coeficiente \( a_k \) (con \( 0 \leq k \leq n \)) corresponde al término de grado \( k \). Así, el coeficiente de mayor grado se encuentra en la última posición del arreglo.

Por ejemplo:

\begin{itemize}
    \item El polinomio \( p(x) = 2x + 1 \) se representa como el arreglo \( A = [1, 2] \), donde \( a_0 = 1 \) y \( a_1 = 2 \).
    \item El polinomio \( p(x) = x^3 - 5x + 1 \) se representa como el arreglo \( A = [1, -5, 0, 1] \), donde \( a_0 = 1 \), \( a_1 = -5 \), \( a_2 = 0 \) y \( a_3 = 1 \).
\end{itemize}

Implemente un método que evalúe el valor del polinomio \( p(x) \) para un valor dado de \( x \). La evaluación del polinomio se realiza mediante la fórmula:

\[
p(x) = a_0 + a_1 x + a_2 x^2 + \dots + a_n x^n
\]

\subsection*{Ejemplos}
\begin{itemize}
    \item Entrada:
    \begin{itemize}
        \item Polinomio: \([1, 2]\) (representa \(p(x) = 2x + 1\))
        \item Valor de \(x\): \(3\)
    \end{itemize}
    Salida: \( p(3) = 7 \)
    
    \item Entrada:
    \begin{itemize}
        \item Polinomio: \([1, -5, 0, 1]\) (representa \(p(x) = x^3 - 5x + 1\))
        \item Valor de \(x\): \(2\)
    \end{itemize}
    Salida: \( p(2) = -1 \)
\end{itemize}