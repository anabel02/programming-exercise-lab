Determine si un número entero positivo es un número de Armstrong. Un número \( n \) se dice que es un \textit{número de Armstrong} (o \textit{número narcisista}) si cumple la siguiente propiedad:

\[
n = \sum_{i=1}^{d} d_i^d,
\]

donde:
\begin{itemize}
    \item \(d\) es el número de dígitos de \(n\),
    \item \(d_i\) representa el \(i\)-ésimo dígito de \(n\).
\end{itemize}

O sea, un número es un número de Armstrong si es igual a la suma de sus dígitos elevados a la cantidad de dígitos del número.

\subsection*{Ejemplos}
\begin{itemize}
    \item Entrada \( n = 153 \)\\
    Salida: \textcolor{blue}{true}\\
    Explicación:
    \(
    \text{Dígitos de } 153: 1, 5, 3, \quad \text{y } 1^3 + 5^3 + 3^3 = 1 + 125 + 27 = 153.
    \)
    
    \item Entrada \( n = 9474 \)\\
    Salida: \textcolor{blue}{true}\\
    Explicación:
    \(
    \text{Dígitos de } 9474: 9, 4, 7, 4, \quad \text{y } 9^4 + 4^4 + 7^4 + 4^4 = 6561 + 256 + 2401 + 256 = 9474.
    \)

    \item Entrada \( n = 9475 \)\\
    Salida: \textcolor{blue}{false}\\
    Explicación:
    \(
    \text{Dígitos de } 9475: 9, 4, 7, 5, \quad \text{y } 9^4 + 4^4 + 7^4 + 5^4 = 6561 + 256 + 2401 + 625 = 9843 \neq 9475.
    \)
    
    \item Entrada \( n = 370 \)\\
    Salida: \textcolor{blue}{true}\\
    Explicación:
    \(
    \text{Dígitos de } 370: 3, 7, 0, \quad \text{y } 3^3 + 7^3 + 0^3 = 27 + 343 + 0 = 370.
    \)
\end{itemize}

\textbf{Propuesta:} Implemente un método que reciba un número entero y calcule su cantidad de dígitos (sin usar la clase \texttt{string}).
