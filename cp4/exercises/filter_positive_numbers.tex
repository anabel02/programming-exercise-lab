\begin{enumerate}[label=\alph*)]
    \item \textbf{Filtrar elementos positivos:} \\
    Implemente un método que reciba un array \(a\) de números enteros y devuelva un nuevo array que contenga únicamente los elementos positivos de \(a\).

    \subsection*{Ejemplos:}
    \begin{itemize}
        \item \textbf{Entrada:} \([1, -2, 3, -4, 5]\) \\
        \textbf{Salida:} \([1, 3, 5]\)
        \item \textbf{Entrada:} \([-10, -5, 0, 2, 4]\) \\
        \textbf{Salida:} \([2, 4]\)
        \item \textbf{Entrada:} \([0, -1, -2]\) \\
        \textbf{Salida:} \([]\)
    \end{itemize}

    \item \textbf{Determinar mayoría positiva:} \\
    Implemente un método que determine si la mayoría de los elementos en un array de números enteros son positivos. El método debe devolver \texttt{true} si más de la mitad de los elementos son positivos, y \texttt{false} en caso contrario.

    \subsection*{Ejemplos:}
    \begin{itemize}
        \item \textbf{Entrada:} \([1, -2, 3, -4, 5]\) \\
        \textbf{Salida:} \textcolor{blue}{true}  (Ya que 3 de 5 elementos son positivos)
        \item \textbf{Entrada:} \([-1, -2, 4, 5]\) \\
        \textbf{Salida:} \textcolor{blue}{false}  (Ya que solo 2 de 4 elementos son positivos)
        \item \textbf{Entrada:} \([1, -1, 0]\) \\
        \textbf{Salida:} \textcolor{blue}{false}  (Ya que solo 1 de 3 elementos es positivo)
    \end{itemize}
\end{enumerate}