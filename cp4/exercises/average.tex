Implemente un método que reciba un array de enteros y devuelva: 
\begin{enumerate}
    \item El promedio de todos los elementos de un array.
    \item La cantidad de elementos que son mayor que el promedio en un array.
\end{enumerate}
\subsection*{Ejemplos}
\begin{itemize}
    \item Entrada: \( \text{array} = [4, 8, 6, 10, 2] \)\\
    Salida: 
    \begin{itemize}
        \item Promedio: 6.0.
        \item Elementos mayores que el promedio: 2.
    \end{itemize}
    Explicación: \(\frac{4 + 8 + 6 + 10 + 2}{5} = 6\).  
    Los elementos mayores que \( 6 \) son \( 8, 10 \), lo que da un total de \( 2 \).

    \item Entrada: \( \text{array} = [1, 1, 1, 1] \)\\
    Salida: 
    \begin{itemize}
        \item Promedio: 1.0.
        \item Elementos mayores que el promedio: 0.
    \end{itemize}
    Explicación: \(\frac{1 + 1 + 1 + 1}{4} = 1\).  
    No hay elementos mayores que \( 1 \), por lo que el resultado es \( 0 \).

    \item Entrada: \( \text{array} = [15, -5, 0, 10, 5] \)\\
    Salida: 
    \begin{itemize}
        \item Promedio: 5.0.
        \item Elementos mayores que el promedio: 2.
    \end{itemize}
    Explicación: \(\frac{15 + (-5) + 0 + 10 + 5}{5} = 5\).  
    Los elementos mayores que \( 5 \) son \( 15, 10 \), lo que da un total de \( 2 \).

    \item Entrada: \( \text{array} = [-10, -20, -30] \)\\
    Salida: 
    \begin{itemize}
        \item Promedio: -20.0.
        \item Elementos mayores que el promedio: 1.
    \end{itemize}
    Explicación: \(\frac{-10 + (-20) + (-30)}{3} = -20\).  
    El único elemento mayor que \( -20 \) es \( -10 \), por lo que el resultado es \( 1 \).
\end{itemize}