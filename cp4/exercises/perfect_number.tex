Determina si un número entero positivo es perfecto. Un número entero positivo \( n \) se denomina \textit{perfecto} si la suma de sus divisores propios (excluyendo a \( n \)) es igual a \( n \).
\[
n \text{ es perfecto} \iff \sum_{\substack{d \mid n \\ d < n}} d = n
\]

donde \( d \mid n \) indica que \( d \) es un divisor de \( n \), es decir, \( n \) es divisible por \( d \). 

\subsection*{Ejemplos:}
\begin{itemize}
    \item Entrada \( n = 28 \)\\
    Salida: \textcolor{blue}{true}\\
    Explicación:
    \(
    \text{Divisores propios de 28: } 1, 2, 4, 7, 14, \quad \text{y} \quad 1 + 2 + 4 + 7 + 14 = 28.
    \)
    
    \item Entrada \( n = 6 \)\\
    Salida: \textcolor{blue}{true}\\
    Explicación:
    \(
    \text{Divisores propiosd de 6: } 1, 2, 3, \quad \text{y } 1 + 2 + 3 = 6.
    \)

    \item Entrada \( n = 12 \)\\
    Salida: \textcolor{blue}{false}\\
    Explicación:
    \(
    \text{Divisores propios de 12: } 1, 2, 3, 4, 6, \quad \text{y } 1 + 2 + 3 + 4 + 6 = 16 \neq 12.
    \)
\end{itemize}
