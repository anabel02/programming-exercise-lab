Dado un array \(a\) y dos posiciones \(i\) y \(j\) (con \(i \leq j\)), implemente un método que devuelva un nuevo array que contenga el rango de elementos del array original comprendido entre las posiciones \(i\) y \(j\) (inclusive). Es decir, el subarray debe estar formado por los elementos desde el índice \(i\) hasta el índice \(j\) del array original.

Si el array \(a\) tiene \(n\) elementos, y \(a = [a_1, a_2, \dots, a_n]\), el subarray que corresponde al rango \([i..j]\) se obtiene seleccionando los elementos \([a_i, a_{i+1}, \dots, a_j]\).

\textbf{Ejemplos:}
\begin{itemize}
    \item Entrada: \(a = [10, 20, 30, 40, 50]\), \(i = 1\), \(j = 3\) \\
    Salida: \([20, 30, 40]\) \\
    En este caso, el subarray es el que va desde el índice \(1\) (valor \(20\)) hasta el índice \(3\) (valor \(40\)), ambos inclusive.

    \item Entrada: \(a = [1, 2, 3, 4, 5]\), \(i = 0\), \(j = 4\) \\
    Salida: \([1, 2, 3, 4, 5]\) \\
    En este caso, el subarray incluye todos los elementos del array original, ya que \(i = 0\) y \(j = 4\).
    
    \item Entrada: \(a = [7, 8, 9, 10, 11]\), \(i = 3\), \(j = 3\) \\
    Salida: \([10]\) \\
    El subarray contiene solo el elemento en la posición \(3\), que es \(10\), ya que \(i = j = 3\).
\end{itemize}