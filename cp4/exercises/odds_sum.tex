Implemente un programa que reciba un entero \(n\) e imprima la suma de los primeros \(n\) números impares.

Formalmente, los números impares son aquellos de la forma \(2k + 1\), donde \(k \in \mathbb{Z}_{\geq 0}\). La suma de los primeros \(n\) números impares se puede expresar como:

\[
S_n = \sum_{k=0}^{n-1} (2k + 1).
\]

\subsection*{Ejemplos}
\begin{itemize}
    \item Entrada: \texttt{1}\\
          Salida: \texttt{La suma de los primeros 1 números impares es 1.}
    \item Entrada: \texttt{3}\\
          Salida: \texttt{La suma de los primeros 3 números impares es 9.}
    \item Entrada: \texttt{5}\\
          Salida: \texttt{La suma de los primeros 5 números impares es 25.}
    \item Entrada: \texttt{7}\\
          Salida: \texttt{La suma de los primeros 7 números impares es 49.}
\end{itemize}
