\begin{enumerate}[label=\alph*)]
    \item Implemente un método que determine si el string \texttt{s} es un palíndromo, es decir, si se lee igual de izquierda a derecha que de derecha a izquierda.
    
    \begin{itemize}
        \item Entrada: \texttt{s = ana}\\
        Salida: \textcolor{blue}{true}

        \item Entrada: \texttt{s = anitalavalatina}\\
        Salida: \textcolor{blue}{true}

        \item Entrada: \texttt{s = palabra}\\
        Salida: \textcolor{blue}{false}
    \end{itemize}

    \item * Implemente un método que compute el menor string \texttt{t} tal que la concatenación de \texttt{s + t} forme un palíndromo.

    \begin{itemize}
        \item Entrada: \texttt{s = race}\\
        Salida: \texttt{car}\\
        Explicación: El string \texttt{''race''} no es un palíndromo. Para que la concatenación \texttt{s + t} sea un palíndromo, debemos agregar \texttt{''car''} al final, formando el palíndromo \texttt{''racecar''}.

        \item Entrada: \texttt{s = aba}\\
        Salida: \texttt{''''}\\
        Explicación: El string \texttt{''aba''} ya es palíndromo.

        \item Entrada: \texttt{s = anan}\\
        Salida: \texttt{a}\\
        Explicación: El string \texttt{''anan''} no es un palíndromo. Al agregar \texttt{''a''} al final, obtenemos el palíndromo \texttt{''anana''}.
    \end{itemize}
\end{enumerate}
