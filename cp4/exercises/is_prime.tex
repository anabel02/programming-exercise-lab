Escribe un programa que determine si un entero es primo o no.
    
Un número entero positivo \( n \) se dice que es \textit{primo} si tiene exactamente dos divisores distintos: \( 1 \) y el propio número \( n \). Es decir, \( n \) es primo si y solo si no existen otros divisores \( d \) tal que \( 1 < d < n \) y \( d \) divide a \( n \). Formalmente, podemos escribir:
\[
n \text{ es primo} \iff  \forall d \in \mathbb{Z}^+ \, \text{se cumple que si} \, d \mid n \text{, entonces } d = 1 \text{ o } d = n.
\]

Donde \( \mathbb{Z}^+ \) representa el conjunto de los números enteros positivos, y \( d \mid n \) denota que \( d \) divide a \( n \), es decir, \( n \) es divisible por \( d \).
\subsection*{Ejemplos}
\begin{itemize}
    \item Entrada: \texttt{10}\\
          Salida: \textcolor{blue}{false}
    \item Entrada: \texttt{29}\\
          Salida: \textcolor{blue}{true}
    \item Entrada: \texttt{15}\\
          Salida: \textcolor{blue}{false}
    \item Entrada: \texttt{31}\\
          Salida: \textcolor{blue}{true}
\end{itemize}