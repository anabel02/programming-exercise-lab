Reciba los coeficientes \(a\), \(b\) y \(c\) (números reales) de una ecuación cuadrática de la forma:

\[
ax^2 + bx + c = 0
\]

y, asumiendo que la ecuación tiene dos soluciones reales, calcule las raíces de la ecuación utilizando la fórmula cuadrática:

\[
x = \frac{-b \pm \sqrt{b^2 - 4ac}}{2a}
\]

Se asegura que el discriminante (\(\Delta = b^2 - 4ac\)) es mayor que 0 para que existan dos soluciones reales.

\subsection*{Ejemplos}
\begin{itemize}
    \item Entrada: \texttt{a = 1, b = -3, c = 2}\\
          Salida: \texttt{x1 = 2, x2 = 1}
    \item Entrada: \texttt{a = 2, b = 5, c = -3}\\
          Salida: \texttt{x1 = 0.5, x2 = -3}
    \item Entrada: \texttt{a = 1, b = -6, c = 8}\\
          Salida: \texttt{x1 = 4, x2 = 2}
\end{itemize}
