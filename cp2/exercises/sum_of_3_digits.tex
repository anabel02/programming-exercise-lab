Escribe un programa que lea un número entero de tres cifras \(n\) de la consola, calcule la suma de sus dígitos y la muestre en la consola.\\
Por ejemplo si \(n = 123\), entonces:
\[
\text{suma} = 1 + 2 + 3
\]
Pista: Usa la división y resto de la división para extraer los dígitos:
\[
\text{Centena} = \frac{n}{100}, \quad \text{Decena} = \frac{(n \% 100)}{10}, \quad \text{Unidad} = n \% 10
\]
\subsection*{Ejemplos}
\begin{itemize}
    \item Entrada: \texttt{n = 123}\\
          Salida: \texttt{Suma = 6}
    \item Entrada: \texttt{n = 456}\\
          Salida: \texttt{Suma = 15}
    \item Entrada: \texttt{n = 789}\\
          Salida: \texttt{Suma = 24}
\end{itemize}