Implemente una clase llamada \textcolor{cyan}{Matrix} para representar el comportamiento de una matriz bidimensional. Una matriz consta de una colección de elementos distribuidos en filas y columnas.

\subsection*{Funcionalidades a implementar}
\begin{itemize}
    \item \textbf{Número de filas}: Obtener el número de filas.
    \item \textbf{Número de columnas}: Obtener el número de columnas.
    \item \textbf{Tamaño}: Obtener el total de elementos de la matriz.
    \item \textbf{Obtener un elemento}: Obtener un elemento específico de la matriz dado su índice (fila, columna).
    \item \textbf{Modificar un elemento}: Modificar un elemento específico de la matriz dado su índice (fila, columna).
    \item \textbf{Suma de matrices}: Realiza la suma de dos matrices y devuelve el resultado como una nueva matriz.
    \item \textbf{Resta de matrices}: Resta una matriz de otra y devuelve el resultado como una nueva matriz.
    \item \textbf{Multiplicación de matrices}: Multiplica dos matrices y devuelve el resultado como una nueva matriz.
    \item \textbf{Transpuesta de una matriz}: Devuelve la transpuesta de la matriz.
    \item \textbf{Multiplicación por un escalar}: Multiplica todos los elementos de la matriz por un escalar y devuelve el resultado como una nueva matriz.
    \item \textbf{Representación en cadena de caracteres}: Devuelve una cadena que representa la matriz en un formato legible.
    \item \textbf{Igualdad}: El resultado será `true` si ambas continen los mismos elementos en las mismas posiciones, de lo contrario debe evaluar `false`.
    \item \textbf{Es simétrica}: Verifica si la matriz es simétrica.
    \item \textbf{Es diagonal}: Verifica si la matriz es una matriz diagonal.
    \item \textbf{Traza de la matriz}: Calcula la traza de la matriz.
    \item * \textbf{Es ortogonal}: Determina si la matriz es ortogonal.
    \item **\textbf{Determinante de una matriz}: Calcula y devuelve el determinante de la matriz (solo aplicable a matrices cuadradas).
    \item **\textbf{Matriz inversa}: Calcula la matriz inversa (si existe).
\end{itemize}
