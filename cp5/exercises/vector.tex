Implemente una clase llamada \textcolor{cyan}{Vector} para representar el comportamiento de un vector en un espacio n-dimensional. Un vector consta de una colección de componentes.

\subsection*{Funcionalidades a implementar}
\begin{itemize}
    \item \textbf{Obtener componente}: Devuelve el valor de un componente en una posición específica del vector.
    \item \textbf{Modificar componente}: Establece el valor de un componente en una posición específica del vector.
    \item \textbf{Tamaño}: Obtener el total de elementos del vector.
    \item \textbf{Norma del vector}: Calcula la norma del vector (norma Euclídea).
    \item \textbf{Suma de vectores}: Realiza la suma de dos vectores y devuelve el resultado como un nuevo vector.
    \item \textbf{Resta de vectores}: Resta un vector de otro y devuelve el resultado como un nuevo vector.
    \item \textbf{Multiplicación por un escalar}: Multiplica todos los componentes del vector por un escalar y devuelve el resultado como un nuevo vector.
    \item \textbf{Producto punto}: Calcula el producto punto (o producto escalar) entre dos vectores.
    \item \textbf{Representación en cadena de caracteres}: Devuelve una cadena que representa el vector en un formato legible.
    \item \textbf{* Ángulo entre vectores}: Calcula el ángulo entre dos vectores.
    \item \textbf{Sistema de ecuaciones}: Implemente un método que dado una matriz A y un vector b resuelva el sistema de ecuaciones lineales $Ax = b$.
\end{itemize}
