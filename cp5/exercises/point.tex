Implementa una clase llamada \textcolor{cyan}{MyPoint} que contenga dos propiedades, \texttt{X} y \texttt{Y}, para almacenar las coordenadas de un punto en el plano.

\subsection*{Funcionalidades a implementar}
\begin{itemize}
    \item \textbf{Distancia entre puntos (\texttt{DistanceBetweenMyPoints})}: 
    Método que reciba dos objetos de tipo \textcolor{cyan}{MyPoint} y devuelva la distancia entre los dos puntos.
\end{itemize}

\subsection*{Métodos adicionales}
Implemente los siguientes métodos de manera externa a la clase \textcolor{cyan}{MyPoint}:
\begin{enumerate}
    \item \textbf{Promedio de distancias (\texttt{AverageDistance})}: 
    Método que recibe un array de objetos \textcolor{cyan}{MyPoint} y un punto de referencia. El método debe calcular y devolver el promedio de las distancias entre cada punto del array y el punto de referencia.

     \item \textbf{Verificar si los puntos están en una recta (\texttt{AreMyPointsCollinear})}: 
    Método que verifica si todos los puntos de un array de \textcolor{cyan}{MyPoint} se encuentran alineados en una única línea recta.

    \item \textbf{Puntos con menor distancia (\texttt{PointsWithMinDistance})}: 
    Método que, dado un array de objetos \textcolor{cyan}{MyPoint}, devuelve los dos puntos que están a la menor distancia entre sí.

    \item * \textbf{Ordenar por cercanía (\texttt{SortByProximity})}: 
    Método que recibe un array de objetos \textcolor{cyan}{MyPoint} y un punto de referencia. El método debe devolver un nuevo array en el que los puntos están ordenados en función de su proximidad al punto de referencia.

    \item * \textbf{Ordenar por cercanía en el lugar (\texttt{SortByProximityInPlace})}: 
    Método que modifica el array original de objetos \textcolor{cyan}{MyPoint}, ordenándolos por su proximidad al punto de referencia, sin generar un nuevo array.
\end{enumerate}
